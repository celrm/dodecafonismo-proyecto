\renewcommand\refname{Referencias}
	\begin{thebibliography}{00}
			
%			\subsubsection*{\autoref{ch:historia}}

			\bibitem{aguirre}
			{ Aguirre, Felipe.} 
			\textit{El concepto de <<disonancia>> en Adorno y en la nueva música} (2019). Consultado en agosto de 2019.
			\\\url{http://www.felipeaguirre.co/blog/2019/1/28/el-concepto-de-disonancia-en-adorno-y-en-la-nueva-musica}
			
			\bibitem{kinney}
			{ Kinney, James P.} 
			\textit{Twelve-tone Serialism: Exploring the Works of Anton Webern},
			University of San Diego.
			Undergraduate Honors Theses.
			(2015)
			\\\url{https://digital.sandiego.edu/honors_theses/1}
			
			\bibitem{diaz}
			{ Díaz de la Fuente, Alicia.} 
			\textit{Estructura y significado en la música serial y aleatoria},
			Universidad Nacional de Educación a Distancia.
			Tesis Doctoral en Filosofía.
			(2005)
			\\\url{https://www2.uned.es/dpto_fim/publicaciones/alicia_1.pdf}		
			
%			\subsubsection*{\autoref{ch:dodecafonismo}}
			
			\bibitem{boulez}
			{ Boulez, Pierre.}
			\textit{Schoenberg is dead},
			The Score.
			(1952)
			\\\url{http://www.ubu.com/papers/Boulez-Schoenberg+Is+Dead.pdf}
			
			\bibitem{dominguez}
			{ Domínguez Romero, Manuel.} 
			\textit{Las Matemáticas en el Serialismo Musical},
			Sigma n.24, 93-98.
			(2004)
			\\\url{http://www.hezkuntza.ejgv.euskadi.eus/r43-573/es/contenidos/informacion/dia6_sigma/es_sigma/adjuntos/sigma_24/6_Serialismo_musical.pdf}

%			\subsubsection*{\autoref{ch:suite}}
			
			\bibitem{ilomaki}
			{ Ilomäki, Tuukka.}
			\textit{On the Similarity of Twelve-Tone Rows},
			Sibelius Academy.
			(2008)
			\\\url{https://helda.helsinki.fi/handle/10138/235041}
			
			\bibitem{hyde}
			{ Hyde, Martha.} Chapter 4: ``Dodecaphonism: Schoenberg'',
			\textit{Models of Musical Analysis: Early Twentieth-century Music},
			Ed. Mark Everist and Jonathan Dunsby.
			Oxford: Blackwell.
			(1993)
			\\\url{https://books.google.es/books?id=JSdVHAAACAAJ}
					
			\bibitem{xiao}
			{ Xiao, June.} 
			\textit{Bach's Influences in the Piano Music of Four 20th Century Composers},
			Indiana University Jacobs School of Music.
			Doctoral Theses in Music.
			(2014)
			\\\url{https://scholarworks.iu.edu/dspace/handle/2022/19212}
			
			\bibitem{clercq}
			{ Clercq, Trevor de.} 
			\textit{A Window into Tonality via the Structure of Schoenberg's ``Musette'' from the Piano Suite, op. 25},
			Theory/Analysis of 20th-Century Music.
			(2006)
			\\\url{http://www.midside.com/pdf/eastman/fall06/th513/schoenberg_op25_analysis.pdf}
						
	\end{thebibliography}