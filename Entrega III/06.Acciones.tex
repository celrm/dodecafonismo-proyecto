\section{M\'AS HERRAMIENTAS MATEM\'ATICAS}\label{ch:acciones}
	\subsection{Acciones de grupos sobre conjuntos}
		Dado un grupo $(G,\ *)$ y un conjunto $X$, la \emph{acci\'on} de $(G,\ *)$ sobre $X$ es una funci\'on $\phi$ que asocia un elemento $g \in G$ y un elemento $x \in X$ -- el par $(g,x)$ -- a otro elemento $g\cdot x$ que tambi\'en pertenece a $X$ \cite{armstrong}. $\phi :(g,x) \to g\cdot x$
	
		La acci\'on $\phi$, expresada mediante la operaci\'on ($\cdot$), debe cumplir dos condiciones:
		\begin{enumerate}
			\item{Para todo $x\in X$, $e\cdot x=x$, siendo $e$ el elemento neutro del grupo.}
		
			\item{Para todo $x\in X$ y para todo par $g,h\in$ G, se debe cumplir que $(g*h)\cdot x=g\cdot (h\cdot x)$. La primera operaci\'on ($*$) es la interna del grupo G, y la segunda operaci\'on ($\cdot$) es la acci\'on.}
		\end{enumerate}
		Como ya se ha visto, las funciones $\{S,\ T,\ V,\ C\}$ forman el grupo di\'edrico ${D}_{n}\times{D}_{n}$, con $n$ la longitud de la serie. Se podr\'a definir entonces la acci\'on $\phi$ de este grupo sobre el conjunto de permutaciones de orden $n$, tal que $\phi(\Psi,\ \sigma)=\Psi\circ\sigma=\Psi(\sigma)=\tau$, con $\Psi\in{D}_{n}\times{D}_{n}$ y $\sigma,\tau\in{S}_n$.
		
		De igual manera, se puede definir el grupo que forman solamente $I$ y $R$, que servir\'a m\'as adelante. Como son dos reflexiones, forman el conocido grupo de Klein \textemdash a partir de ahora denotado por $\Xi$, con elementos $Id$, $I$, $R$ e $IR$.

	\subsection{\'Orbitas y estabilizadores}	
		Dada una acci\'on de $(G,\ *)$ sobre $X$, la \emph{\'orbita} de un determinado elemento $x_0\in X$ es el subconjunto de elementos $x$ de $X$ que pueden ser alcanzados desde $x_0$ mediante alg\'un $g_0\in G$. Es decir, todos los $x$ para los que existe un $g_0$ que al actuar sobre $x_0$ da $x$. Trivialmente, $x_0\in Orb(x_0)$ ya que $e\cdot x_0=x_0$.
		\[Orb(x_0)=\{x\in X :\ \exists \ g_0\in {G},\ g_0\cdot x_0 =x\}\]
	
		Por ejemplo, dada una permutaci\'on $\sigma$, todas las permutaciones a las que se llega desde $\sigma$ mediante alg\'un $\Psi\in{D}_{n}\times{D}_{n}$
		conforman la \'orbita de $\sigma$. Por definici\'on, las series a las que se puede llegar desde una serie original conforman su espectro serial, por lo que \textbf{la \'orbita es en realidad el espectro serial}.
	
		Para el mismo $x_0$ se define su \emph{estabilizador} como el conjunto de elementos $g\in G$ que fijan $x_0$, es decir, que mandan $x_0$ a s\'i mismo. Mientras que una \'orbita es un subconjunto de $X$, un estabilizador es un subgrupo de $G$. Trivialmente, $e\in Stab(x)\ \forall x\in X$, porque el elemento identidad fija cualquier otro elemento por definici\'on.
		\[Stab(x_0)=\{g\in {G}\ :\ g\cdot x_0 =x_0 \}\]
	
		Si cada $g\in G$ llevara a $x_0$ a un $x$ distinto, el n\'umero de elementos de $Orb(x_0)$ ser\'ia igual al n\'umero de elementos de $G$. Sin embargo, si un elemento $g_0\in G$ fija $x_0$, entonces no dar\'a nuevos elementos en la \'orbita de $x$. Por tanto, intuitivamente el tama\~no de la \'orbita disminuye. De hecho,  el teorema de \'Orbita--Estabilizador dice que el tama\~no de una \'orbita ($|Orb(x_0)|$) ser\'a el tama\~no de $G$ $(|G|)$ entre el n\'umero de elementos que fijan $x_0$; es decir, el tama\~no de su estabilizador ($|Stab(x_0)|$). Adem\'as, es cierto para todo $x\in X$.
		\[|Orb(x)|=\frac{|{G}|}{|Stab(x)|}\mbox{, o lo que es lo mismo, }|{G}|=|Orb(x)||Stab(x)|\]
		
		\def\arraystretch{1.5}
		Este teorema implica que los tama\~nos de cada \'orbita y cada estabilizador son divisores del tama\~no del grupo. Por ejemplo, como el tama\~no del grupo $\Xi$ es 4, cualquier estabilizador y cualquier \'orbita tendr\'an tama\~no 1, 2 o 4. En concreto, como $Id$ est\'a siempre en el estabilizador, para  todo $\sigma$ ser\'a de una de estas formas:
		\[\begin{array}{*{5}c}
		|Stab|=1&&&\{{Id}\}&\\
		\hline|Stab|=2&&\{Id,\ I\}&\{Id,\ R\}&\{Id,\ IR\}\\
		\hline|Stab|=4&&&\{Id,\ I,\ R,\ IR\}&\\
		\end{array}\]
		
		\def\arraystretch{1}
		Una serie $\sigma$ sin simetr\'ias tendr\'a una serie distinta para cada una de sus transformaciones. Por tanto, su \'orbita ser\'a $\{\sigma,\ R(\sigma),\ I(\sigma),\ IR(\sigma)\}$ y su estabilizador ser\'a solamente \{$Id$\}. Cumple entonces el teorema: $4\cdot 1 = 4$.
	
	\subsection{El lema de Burnside}
		\label{burnside}
		Las \'orbitas, que son subconjuntos de $X$, forman una \emph{partici\'on} de $X$. Esto significa que son subconjuntos disjuntos: ning\'un $x$ puede estar en dos \'orbitas distintas. Interesa entonces saber cu\'antos subconjuntos hay; es decir, el n\'umero de \'orbitas ($\#Orb$). El lema de Burnside afirma que se pueden calcular as\'i:
		\[\#Orb=\frac{1}{|{G}|}\sum_{x\in{X}}|{Stab}(x)|\]	
		Se prueba de esta forma: por el teorema de \'Orbita--Estabilizador, $|{Stab}(x)|=\frac{|{G}|}{|Orb(x)|}$, por lo que la parte derecha se puede expresar as\'i:
		\[\frac{1}{|{G}|}\sum_{x\in X}|{Stab}(x)|=
		\frac{1}{|{G}|}\sum_{x\in X}\frac{|{G}|}{|Orb(x)|}=
		\sum_{x\in X}\frac{1}{|Orb(x)|}\]

		Como las \'orbitas forman una partici\'on de $X$, la suma sobre todo el conjunto $X$ puede ser dividida en sumas separadas para cada \'orbita. Adem\'as, si por cada elemento de una \'orbita se suma el inverso del n\'umero de elementos de la \'orbita, esa suma dar\'a uno. Solo queda ahora sumar uno por cada \'orbita.
		\[\sum_{x\in X}\frac{1}{|Orb(x)|}=\sum_{{O}\in\mbox{\'Orbitas}}\left(\sum_{x\in{O}}\frac{1}{|{O}|}\right)=\sum_{O\in\mbox{\'Orbitas}}1=\#Orb \qquad\square\]	
	
		Este lema permite calcular el n\'umero de posibles espectros seriales distintos, ya que el espectro de una serie es igual al espectro de sus series transformadas. Un compositor serialista debe entonces escoger no una serie original, sino el espectro con el que construir la obra. O, m\'as bien, si escoge una serie original est\'a escogiendo el mismo material que si escogiera otra serie de ese mismo espectro.