%	\chapter{RESULTADOS DE LAS MODIFICACIONES}
\section{MODIFICACIÓN DE PARTITURAS SERIALISTAS}\label{ch:musescore}
	\subsection{Obras modificadas}
	
	Ahora se describirán las obras que pasarán por la modificación. Para abarcar distintos estilos compositivos y hacer este estudio más riguroso, se han escogido obras de los tres principales compositores dodecafónicos: Schoenberg, Berg y Webern.
	
	Sin embargo, no se han escogido obras de compositores posteriores ni serialistas integrales. Uno de los motivos es porque interesa en este estudio la relación entre los sonidos: no se modifican más que las alturas de las notas, y por tanto no importa el resto de elementos musicales. Que estén compuestos serialmente no afecta a las conclusiones de este experimento.
	
	Por otro lado, los compositores posteriores a Schoenberg todavía no han pasado al dominio público. Eso impide, por desgracia, que se pueda trabajar libremente con su música.
	
	Por último, el hecho de que cada nota tenga su propia dinámica, su propia articulación o su propio timbre hace de las obras serialistas integrales que sean difíciles de manipular. Además, como los audios están hechos mediante ordenador y no con intérpretes reales, la calidad y la intención musical de estas partituras tan complicadas nunca podrían plasmarse a la perfección.
	
	La primera obra que pasará por el algoritmo de modificación serial es la \textit{Suite para piano}, Op. 25 de Schoenberg.
	
	{Su serie original\footnote{\url{http://www.ccarh.org/publications/data/humdrum/tonerow/files/schoenberg/schoenberg04.pc.krn}} es: \drow{4,5,7,1,6,3,8,2,11,0,9,10}}
	\ddiagram{4,5,7,1,6,3,8,2,11,0,9,10}
	
	La segunda obra es un arreglo para soprano y piano de una de las arias más destacadas de la segunda ópera de Alban Berg, \textit{Lulu}. El libreto de la obra está basado en dos tragedias de Frank Wedekind: ``El espíritu de la tierra'' y ``La Caja de Pandora''.
	
	El aria, llamada \textit{Lied der Lulu}, es parte de una dramática disputa entre Lulu y su marido por las infidelidades de ella, que acaba con el homicidio accidental de él.
	
	{La serie de Lulu\footnote{\href{http://www.ccarh.org/publications/data/humdrum/tonerow/files/berg/berg10.pc.krn}{ \texttt{\textasciitilde/berg/berg10.pc.krn} }} es:
		\drow{0,4,5,2,7,9,6,8,11,10,3,1}}
	\ddiagram{0,4,5,2,7,9,6,8,11,10,3,1}
	
	La tercera, \textit{Der Wein}, es un aria de concierto para soprano y orquesta compuesta por Berg en 1929. La letra es una traducción al alemán de los tres poemas de Charles Baudelaire ``\textit{Le Vin}''. Son una celebración del vino y de la felicidad de quienes lo toman.
	
	La pieza evoca aires tonales, ya que su serie está compuesta por la escala de Re menor y la de Sol bemol Mayor. Es una pieza simétrica que evoca también la forma sonata, con los tres poemas formando una exposición, un desarrollo y una recapitulación.
	
	{Su serie original\footnote{\href{http://www.ccarh.org/publications/data/humdrum/tonerow/files/berg/berg09.pc.krn}{ \texttt{\textasciitilde/berg/berg09.pc.krn} }} es:        	
		\drow{2,4,5,7,9,10,1,6,8,0,11,3}}
	\ddiagram{2,4,5,7,9,10,1,6,8,0,11,3}
	
	La cuarta, de 1936, es la única obra publicada de Webern para piano solo: \textit{Variationen für Klavier}, Op. 27, y se compone de tres movimientos: \textit{Sehr mässig}, \textit{Sehr schnell} y \textit{Ruhig fliessend}.
	
	{Su serie original\footnote{\href{http://www.ccarh.org/publications/data/humdrum/tonerow/files/webern/webern17.pc.krn}{ \texttt{\textasciitilde/webern/webern17.pc.krn} }} es:
		\drow{3,11,10,2,1,0,6,4,7,5,9,8}}
	\ddiagram{3,11,10,2,1,0,6,4,7,5,9,8}
	
	Por último, \textit{3 Lieder}, Op. 18, compuesta por Webern en 1925, es un tríptico de Lieder para voz, clarinete y guitarra. Los Lieder, junto con sus respectivas series, son:
	
	\textit{Schatzerl Klein}\footnote{\href{http://www.ccarh.org/publications/data/humdrum/tonerow/files/webern/webern06.pc.krn}{ \texttt{\textasciitilde/webern/webern06.pc.krn} }}: \hfill
	\drow{0,11,5,8,10,9,3,4,1,7,2,6}
	\ddiagram{0,11,5,8,10,9,3,4,1,7,2,6}
	
	\textit{Erlosung}\footnote{\href{http://www.ccarh.org/publications/data/humdrum/tonerow/files/webern/webern07.pc.krn}{ \texttt{\textasciitilde/webern/webern07.pc.krn} }}: \hfill
	\drow{6,9,5,8,4,7,3,11,2,10,1,0}
	\ddiagram{6,9,5,8,4,7,3,11,2,10,1,0}
	
	\textit{Ave, Regina Coelorum}\footnote{\href{http://www.ccarh.org/publications/data/humdrum/tonerow/files/webern/webern08.pc.krn}{ \texttt{\textasciitilde/webern/webern08.pc.krn} }}: \hfill
	\drow{4,3,7,6,5,11,10,2,1,0,9,8}
	\ddiagram{4,3,7,6,5,11,10,2,1,0,9,8}
    
    % 10.plugin de musescore, conclusiones
    \subsection{Página de modificaciones y plugin}
	
	La primera vez que realicé este experimento tuve que modificar nota a nota, a mano, la partitura que había escogido. Por este motivo decidí crear estas herramientas, que evitan ese trabajo tedioso y mecánico, pero también sirven para otros propósitos. Por ejemplo, para cambiar una partitura de mayor a menor, o viceversa.
	
    He creado una página interactiva que transforma cada nota de una partitura a la nota requerida. Está escrita en Elm y el código puede encontrarse en \url{https://gitlab.com/dodecafonismo/modificaciones}.  	
    	
   	En el enlace \url{https://modificaciones.netlify.com/} está la aplicación web. Sus instrucciones de uso se encuentran al final de la página.
    
    \subsection{Obra de Schoenberg} % TODO
%   
%   	\[\left.\begin{matrix}
%   	\text{Cromática:}&0&1&2&3&4&5&6&7&8&9&10&11\\
%   	\text{Pentatónica:}&0&0&2&2&4&4&7&7&7&9&9&0\\
%   	\text{Intervalos:}&&2&&2&&3&&&2&&3&\\
%   	\end{matrix}\right.\]
%   	
%   	Se puede observar que, ya que 5 no es divisor de 12, no hay una repartición equitativa, por lo que en cada serie habrá notas que aparezcan más que otras. En esta función, las notas repetidas son el Do (nota 0) y el Sol (nota 7). Además, estas notas forman el bordón de la Musette, por lo que tendrá aspecto sonoro de Do Mayor. %En el Anexo , página  se encuentra la partitura de la modificación pentatónica, sin incluir las dinámicas por cuestión de simplificación, y en la pista 2 se encuentra la grabación de la misma, creada con el programa Musescore.
%   	
%   	%Un estudio ulterior muy interesante consistiría en probar con otras funciones que repitieran notas diferentes, o probar con otras escalas como la hexafónica (de tonos enteros) o la heptafónica (las escalas tonales), y sus respectivas funciones posibles, o incluso aplicarlo a diversas obras. La extensión de mi investigación no puede abarcar ese trabajo, además de que se necesitaría un programa que aplicara automáticamente las funciones a la partitura en vez de tener que cambiar cada nota manualmente.
%   	
%   	%Sin embargo, he hecho una prueba sobre la primera sección de la Musette con una única función de la escala hexafónica y otra de la heptafónica, para así justificar mi elección de la escala pentatónica como la mejor entre las tres.
%   	
%   	Con la escala hexafónica habría una repartición equitativa en la función, por lo que la obra seguiría siendo estrictamente serialista y ninguna nota sobresaldría. El problema de esta escala es que tampoco suena natural al oído.%, como se puede comprobar en la pista 3 (partitura en el Anexo , página ), que es la modificación hexafónica de la primera sección de la Musette con la siguiente función:
%   	%\[\left.\begin{matrix}\text{Escala dodecafónica:}&0&1&2&3&4&5&6&7&8&9&10&11\\\text{Escala hexafónica:}&0&0&2&2&4&4&6&6&8&8&10&10\\\end{matrix}\right.\]
%   	
%   	Por último, la escala heptafónica tiene el problema de contener dos intervalos de semitono, por lo que la obra modificada suena también disonante. %Esto se muestra en la pista 4 (partitura en el Anexo , página ), que es la modificación heptafónica de la primera sección de la Musette con la siguiente función:
%   	%\[\left.\begin{matrix}\text{Escala dodecafónica:}&0&1&2&3&4&5&6&7&8&9&10&11\\\text{Escala heptafónica:}&0&0&2&2&4&5&5&7&7&9&9&11\\\end{matrix}\right.\]
%   	
%   	\[\left.\begin{matrix}
%   	\text{Cromática:}&0&1&2&3&4&5&6&7&8&9&10&11\\
%   	\text{Tonos enteros (6):}&0&0&2&2&4&4&6&6&8&8&10&10\\
%   	\text{Intervalos:}&&2&&2&&2&&2&&2&&2\\
%   	\end{matrix}\right.\]
%   	\[\left.\begin{matrix}
%   	\text{Cromática:}&0&1&2&3&4&5&6&7&8&9&10&11\\
%   	\text{Diatónica en Do (7):}&0&0&2&2&4&5&5&7&7&9&9&11\\
%   	\text{Intervalos:}&&2&&2&1&&2&&2&&2&1\\
%   	\end{matrix}\right.\]        
%   	\[\left.\begin{matrix}
%   	\text{Cromática:}&0&1&2&3&4&5&6&7&8&9&10&11\\
%   	\text{Octotónica (8):}&0&0&2&3&3&5&6&6&8&9&9&11\\
%   	\text{Intervalos:}&&2&1&&2&1&&2&1&&2&1\\
%   	\end{matrix}\right.\]
   	
    \subsection{Obras de Berg}
    \subsection{Obras de Webern}
%    \subsection{Conclusiones}