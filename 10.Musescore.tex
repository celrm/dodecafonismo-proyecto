	\chapter{RESULTADOS DE LAS MODIFICACIONES}
    
    % 10.plugin de musescore, conclusiones
    \section{Página de modificaciones y plugin}
   
    He creado una página interactiva que transforma cada nota de una partitura a la nota requerida. Está escrita en Elm y el código puede encontrarse en \textit{https://gitlab.com/dodecafonismo/modificaciones}.
   
    \begin{wrapfigure}{l}{0.2\textwidth}
    	\vspace{-0.5cm}
    	\qrcode{https://modificaciones.netlify.com/}
    	\vspace{-1.5cm}
    \end{wrapfigure} Este es el enlace de la aplicación web. Sus instrucciones de uso se encuentran al final de la página. El enlace es \textit{https://modificaciones.netlify.com/}.
    
    \section{Obra de Schoenberg}
   
   	Tomando la música debussiana y las músicas orientales como referencia, he escogido la escala pentatónica para aplicarla a la Musette de la Suite para piano Op. 25 de Schoenberg. Para relacionar la escala dodecafónica con la nueva escala, se debe crear una función que relacione las notas de ambos conjuntos. Yo he tomado esta función:
   	$$\left.\begin{matrix}\text{Escala dodecafónica:}&0&1&2&3&4&5&6&7&8&9&10&11\\\text{Escala pentatónica:}&0&0&2&2&4&4&7&7&7&9&9&0\\\end{matrix}\right.$$
   	$$\left.\begin{matrix}\text{Do}&\text{Do\#}&\text{Re}&\text{Re\#}&\text{Mi}&\text{Fa}&\text{Fa\#}&\text{Sol}&\text{Sol\#}&\text{La}&\text{La\#}&\text{Si}\\\text{Do}&\text{Do}&\text{Re}&\text{Re}&\text{Mi}&\text{Mi}&\text{Sol}&\text{Sol}&\text{Sol}&\text{La}&\text{La}&\text{Do}\\\end{matrix}\right.$$
   	
   	Se puede observar que, ya que 5 no es divisor de 12, no hay una repartición equitativa, por lo que en cada serie habrá notas que aparezcan más que otras. En mi función, las notas repetidas son el Do (nota 0) y el Sol (nota 7). Además, estas notas forman el bordón de la Musette, por lo que tendrá aspecto sonoro de Do Mayor. En el Anexo , página  se encuentra la partitura de la modificación pentatónica, sin incluir las dinámicas por cuestión de simplificación, y en la pista 2 se encuentra la grabación de la misma, creada con el programa Musescore.
   	
   	Un estudio ulterior muy interesante consistiría en probar con otras funciones que repitieran notas diferentes, o probar con otras escalas como la hexafónica (de tonos enteros) o la heptafónica (las escalas tonales), y sus respectivas funciones posibles, o incluso aplicarlo a diversas obras. La extensión de mi investigación no puede abarcar ese trabajo, además de que se necesitaría un programa que aplicara automáticamente las funciones a la partitura en vez de tener que cambiar cada nota manualmente.
   	
   	Sin embargo, he hecho una prueba sobre la primera sección de la Musette con una única función de la escala hexafónica y otra de la heptafónica, para así justificar mi elección de la escala pentatónica como la mejor entre las tres.
   	
   	Con la escala hexafónica habría una repartición equitativa en la función, por lo que la obra seguiría siendo estrictamente serialista y ninguna nota sobresaldría. El problema de esta escala es que tampoco suena natural al oído, como se puede comprobar en la pista 3 (partitura en el Anexo , página ), que es la modificación hexafónica de la primera sección de la Musette con la siguiente función:
   	$$\left.\begin{matrix}\text{Escala dodecafónica:}&0&1&2&3&4&5&6&7&8&9&10&11\\\text{Escala hexafónica:}&0&0&2&2&4&4&6&6&8&8&10&10\\\end{matrix}\right.$$
   	$$\left.\begin{matrix}\text{Do}&\text{Do\#}&\text{Re}&\text{Re\#}&\text{Mi}&\text{Fa}&\text{Fa\#}&\text{Sol}&\text{Sol\#}&\text{La}&\text{La\#}&\text{Si}\\\text{Do}&\text{Do}&\text{Re}&\text{Re}&\text{Mi}&\text{Mi}&\text{Fa\#}&\text{Fa\#}&\text{Sol\#}&\text{Sol\#}&\text{La\#}&\text{La\#}\\\end{matrix}\right.$$
   	
   	Por último, la escala heptafónica tiene el problema de contener dos intervalos de semitono, por lo que la obra modificada suena también disonante. Esto se muestra en la pista 4 (partitura en el Anexo , página ), que es la modificación heptafónica de la primera sección de la Musette con la siguiente función:
   	$$\left.\begin{matrix}\text{Escala dodecafónica:}&0&1&2&3&4&5&6&7&8&9&10&11\\\text{Escala heptafónica:}&0&0&2&2&4&5&5&7&7&9&9&11\\\end{matrix}\right.$$
   	$$\left.\begin{matrix}\text{Do}&\text{Do\#}&\text{Re}&\text{Re\#}&\text{Mi}&\text{Fa}&\text{Fa\#}&\text{Sol}&\text{Sol\#}&\text{La}&\text{La\#}&\text{Si}\\\text{Do}&\text{Do}&\text{Re}&\text{Re}&\text{Mi}&\text{Fa}&\text{Fa}&\text{Sol}&\text{Sol}&\text{La}&\text{La}&\text{Si}\\\end{matrix}\right.$$	
   
    \section{Obras de Berg}
    \section{Obras de Webern}
    \section{Conclusiones}