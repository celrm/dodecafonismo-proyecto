\documentclass[]{article}

\usepackage{parskip}
\usepackage{microtype}
%\usepackage[T1]{fontenc} % dejar solo para html
%\usepackage[utf8]{inputenc}
\usepackage[latin1]{inputenc}
\usepackage[a4paper]{geometry}

\usepackage{graphicx}
\graphicspath{{Fotos/}}

\usepackage{cancel}

\usepackage{amsmath}
\usepackage{amsfonts}
\usepackage{harmony}
\usepackage{wasysym}

\input{ddphonism.sty}

\usepackage{multicol}
\PassOptionsToPackage{hyphens}{url}
\usepackage{hyperref}

\newcommand{\cita}[2][Tomado de]{{\footnotesize (#1 #2)}}

\title{Dodecafonismo y matem�ticas - I}
\author{Celia Rubio}
\date{Septiembre de 2019}                                           % Activate to display a given date or no date

\begin{document}
\maketitle
	
Introducci\'on de Paco:

En las siguientes entregas de Divulgamat tendremos a una autora invitada, Celia Rubio, quien ha escrito una serie sobre el dodecafonismo y matem�ticas. Celia Rubio est� cursando el doble grado de Matem�ticas e Inform�tica en la Universidad Complutense de Madrid y tiene estudios de m�sica en el Conservatorio de Madrid (su instrumento es la flauta de pico y el canto).  
	
	
\section{Introducci�n}

	Este artículo es el primero de una colección sobre el serialismo musical y sobre las matemáticas que lo rodean. Las músicas serialistas se basan en la repetición de secuencias de elementos musicales. Es decir, un compositor serialista cogerá una secuencia de notas, dinámicas o timbres y la usará como el bloque constructivo de su obra. Además puede serializar más de un tipo de elemento, o incluso pretender serializar todo lo que pueda. Estas músicas se pueden describir matemáticamente mediante permutaciones y grupos.
	
	En estas estructuras se centrará esta serie de artículos, y más específicamente en el \textit{dodecafonismo}, el primer sistema compositivo serialista. En este primer artículo hablaremos sobre sus \hyperref[ch:historia]{orígenes} y sobre los \hyperref[ch:dodecafonismo]{postulados} matemáticos que lo definieron, y \hyperref[ch:suite]{analizaremos} una obra como ejemplo. Más adelante generalizaremos las definiciones dodecafónicas matemáticamente, descubriremos la historia del \textit{serialismo integral}, y contaremos el número de posibles series distintas -- o, más bien, de \textit{espectros seriales} -- que un compositor puede utilizar. En definitiva, será un compendio de lo que se puede estudiar sobre el serialismo.
	
	Los textos irán dirigidos tanto a matemáticos como a músicos; en todas las entregas habrá secciones más matemáticas y secciones más musicales o históricas. Las matemáticas serán avanzadas, pero siempre se definirá todo lo que se utilice y se probará todo lo que se afirme. Algunas de las definiciones matemáticas más comunes se encuentran en el apartado \ref{ch:permutaciones}.
	
	\section{Introducci�n hist�rica del dodecafonismo}\label{ch:historia}

En esta secci�n describiremos cu�l fue el ambiente hist�rico y musical en el que se cultiv� el primer modelo de serialismo musical: el \textbf{dodecafonismo}. A trav�s de su historia analizaremos por qu� el serialismo no fue una decisi�n aleatoria ni espont�nea, sino que surgi� de una necesidad est�tica de aquel periodo.
%
Vamos a comenzar con una breve cr�nica de la disonancia, tras lo cual describiremos las fases por las que el creador del dodecafonismo, Arnold Schoenberg, tuvo que pasar antes de concebirlo. %La disonancia desempe�� un papel fundamental en la g�nesis del dodecafonismo, como veremos enseguida.

\subsection{La historia de la disonancia}
La disonancia siempre ha formado parte de la experiencia musical. Con la m�sica ha venido siempre emparejada la disonancia, mano a mano, como instrumento de contraste, confrontaci�n y ruptura, pero tambi�n como elemento constructivo del discurso musical.

En la Antigua Grecia, la armon�a musical se consideraba unida al resto del universo. La rotaci�n de los astros emit�a sonidos arm�nicos, y era la armon�a la que apaciguaba el alma. Pero �qu� era la armon�a sino la uni�n de consonancia y disonancia? Como dijo Arist�teles:
\begin{quote}
\emph{El alma es armon�a porque la armon�a es mezcla y s�ntesis de contrarios, y de contrarios precisamente est� compuesto el cuerpo.}\\
{\footnotesize (Tomado de \cite{mha} J. de Aixquivel, \textit{Memorias de Historia Antigua}, 1989.)}
\end{quote}

Es bien sabido que la Escuela de Pit�goras, con su estudio sobre proporciones entre notas, buscaba encontrar cu�les eran los intervalos m�s consonantes: eran aquellos cuya proporci�n formaba una relaci�n sencilla. El intervalo de octava era consonante porque su ratio era de 2:1, y de igual manera ocurr�a con los intervalos de quinta (3:2) y cuarta (4:3), a los que Arist�xeno comenz� a llamar \textit{s�mph$\bar{o}$nos} ($\sigma\acute{\upsilon}\mu\varphi\omega\nu o\varsigma$) \cite{cons}. En cambio, a los intervalos no tan sencillos se los llamaba \textit{di�ph$\bar{o}$nos} ($\delta\iota\acute{\alpha}\varphi\omega\nu o\varsigma$), y fue entonces cuando se le dio nombre a la disonancia.

Ya en la Edad Media, la polifon�a fue forjando normas sobre su uso. La primera regla compositiva de la m�sica occidental \textemdash seg�n Knud Jeppesen~\cite{jeppesen}\textemdash~fue la \emph{regla franconiana}, que expresaba que las disonancias deb�an ocurrir en la parte d�bil del comp�s, mientras que las consonancias en la parte fuerte. Es as� como los compositores trenzaban consonancia y disonancia al tejer los hilos de la m�sica.

Poco a poco la disonancia pas� a ser usada como floritura mel�dica: en notas de paso, apoyaturas o retardos, entre otras. Esta funci�n mel�dica fue impregnando el contrapunto hasta llegar a ser pieza clave en la continuidad y el enlace de las voces. Adquiri� entonces una nueva funci�n contrapunt�stica. �Qui�n no se ha deleitado al escuchar una disonancia \textit{bachiana}?

Pero la disonancia estaba a�n circunscrita a la tonalidad reinante. No fue hasta la introducci�n de acordes extra�os que la disonancia pas� a ser el centro del inter�s musical, y fue \textit{in crescendo} apropi�ndose del foco de atenci�n hasta llegar a ser m�s valiosa a�n que la consonancia. Para ello hubo que esperar hasta el siglo XIX, que fue testigo de un asombroso desarrollo del sistema arm�nico que acab� por quebrantar todas las concepciones musicales anteriores.

Para m�s informaci�n sobre la disonancia y su fascinante historia, recomendamos al lector el texto de Felipe Aguirre~\cite{aguirre}.

\subsection{Wagner, Mahler y la emancipaci�n de la disonancia}
Aunque las posibilidades que promet�a la tonalidad parec�an inagotables, sus l�mites comenzaron a percibirse hacia finales del siglo XIX. En palabras de Arnold Schoenberg:
\begin{quote}
\emph{El o�do se fue familiarizando gradualmente con gran n�mero de disonancias, hasta que lleg� a perder el miedo a su efecto perturbador.}\\
{\footnotesize Mencionado en \cite{scho} \emph{Composition with twelve tones}, de \emph{Style and Idea}, 1950.}
\end{quote}

Esta �poca culmin� con los dramas musicales de Richard Wagner, en los que todos los elementos de la obra estaban detalladamente estudiados por el compositor. A este concepto lo llamaba \emph{Gesamtkunstwerk} (``obra de arte total") \textemdash mencionado en \cite{wagner} \emph{Oper und Drama}, 1851\textemdash, ya que se aseguraba personalmente de que en sus �peras las artes esc�nicas, musicales, po�ticas y visuales se combinaran entre s� a la perfecci�n.

\begin{figure}[h]
\begin{center}
\includegraphics[width=5cm]{Richard_Wagner.jpg}\\
\caption{Richard Wagner (1813\textemdash1883); figura tomada de \href{https://www.nationalgeographic.com.es/historia/actualidad/richard-wagner-nacimiento-y-muerte_7014}{National Geographic}.}
\end{center}
\end{figure}

La idea del \emph{Gesamtkunstwerk} la desarroll� alrededor de 1850, y la plasm� en su totalidad en su ciclo de cuatro �peras \href{https://www.youtube.com/watch?v=1PBhlPeTJ_g}{\textit{Der Ring des Nibelungen}}, estrenado en 1876. Wagner control� y cre� cada aspecto de la tetralog�a, desde la m�sica hasta el libreto, el vestuario y la escenograf�a. Incluso mand� crear su propia sala de conciertos en Bayreuth, el \emph{Festspielhaus}, para que el escenario se adecuara a sus ideas sobre el pensamiento y la cultura musical; v�ase~\cite{kinney} para m�s detalles.

As�, a ojos de compositores posteriores, se hab�an agotado todas las posibilidades de la m�sica tonal, y quiz�s ya hab�a comenzado el viraje hacia el predominio de la disonancia con su abundante uso del cromatismo, como en el famoso primer acorde del drama musical \href{https://www.youtube.com/watch?v=SF4zN-Okonc}{\textit{Tristan und Isolde}} (1865). Consta de las notas fa-si-$\mbox{re\#}$-$\mbox{sol\#}$, y sus intervalos desde el fa son una cuarta aumentada, una sexta aumentada y una novena aumentada.

Despu�s de Wagner, otros compositores tambi�n estuvieron a las puertas de emancipar la disonancia, de desatarla de las ataduras que impon�a la tonalidad. Por ejemplo, el gran compositor Gustav Mahler consegu�a reflejar en sus sinfon�as dos realidades paralelas: tanto la delicada fragilidad de la tradici�n anterior como la inminencia de su ruptura. El ejemplo m�s claro es el \href{https://www.youtube.com/watch?v=vHyV8noUXC0}{Adagio} de su D�cima Sinfon�a, que contiene una disonancia con once de las doce notas de la escala crom�tica. Y es que, sin lugar a dudas, ya se preve�a que la tonalidad iba a reemplazarse.

\begin{figure}[h]
\begin{center}
\includegraphics[width=5cm]{Gustav_Mahler.jpg}\\
\caption{Gustav Mahler (1860\textemdash1911); figura tomada de \href{http://www.planethugill.com/2018/09/mahler-distilled-iain-farrington-and.html}{Planet Hugill}.}
\end{center}
\end{figure}

Siguiendo la concepci�n del progreso como un camino ascendente, el paso siguiente para la composici�n musical deb�a consistir en deshacerse progresivamente de la tonalidad y desarrollar la ``{emancipaci�n de la disonancia}"  \textemdash mencionado tambi�n en \cite{scho} \emph{Composition with twelve tones}\textemdash. As�, en el marco expresionista del cambio de siglo, fue como Arnold Schoenberg ide� sus teor�as del pensamiento musical, y �stas dieron paso a la creaci�n de la atonalidad.

\subsection{Hacia el atonalismo de Schoenberg}
Fuertemente influido por Wagner y Mahler desde su adolescencia, Schoenberg comenz� componiendo al estilo posrom�ntico de su �poca, llevando el cromatismo y la orquestaci�n hasta el extremo. Sin embargo, y no espont�neamente, empez� a buscar en sus composiciones que cada sonido tuviera valor por s� mismo, un valor independiente de su funcionalidad tonal.

\begin{figure}[h]
\begin{center}
\includegraphics[width=5cm]{Arnold_Schoenberg.jpg}\\
\caption{Arnold Schoenberg (1874\textemdash1951); figura tomada de \href{http://es.nextews.com/59ce14c7/}{Nextews}.}
\end{center}
\end{figure}

Para �l, la m�sica no estaba intr�nsecamente dirigida a una t�nica. En las progresiones, lo importante era el paso de un acorde a otro, y no hacia d�nde se dirig�an estos. Adem�s, �l opinaba que se deb�an poder utilizar las notas de los modos eclesi�sticos libremente, por lo que consideraba las notas no diat�nicas tan v�lidas como las diat�nicas. Esto hac�a imposible distinguir unas de otras, y apenas se pod�a identificar la t�nica. De esta, y de otras muchas formas, Schoenberg consegu�a que la jerarqu�a tonal quedara desestabilizada~\cite{kinney}.

De esta �poca es su primera obra importante, \href{https://www.youtube.com/watch?v=vqODySSxYpc}{\emph{Verkl�rte Nacht}} ({\it Noche transfigurada}), Op. 4. Compuesto en 1899, este sexteto de cuerdas est� inspirado por el poema hom�nimo de Richard Dehmel. La m�sica, seg�n su autor, expresa el paseo de un hombre y una mujer en medio de la naturaleza.  Aunque en la obra a�n prevalece la armon�a tradicional basada en acordes, Schoenberg sit�a al oyente en un terreno de indefinici�n tonal, no s�lo en el plano arm�nico sino tambi�n en el mel�dico. Adem�s, hace uso del acorde de novena invertido, inexistente hasta entonces y, por tanto, rechazado por la cr�tica~\cite{diaz}.

Tras pasar por la etapa tonal post-rom�ntica, y debido a su convicci�n en la inexorabilidad de la evoluci�n de la m�sica hacia el cromatismo total, en 1908 Schoenberg se deslig� de la tonalidad completamente con el ciclo de canciones \href{https://www.youtube.com/watch?v=3iXsKhaZB2Q}{\emph{Das Buch der H�ngenden G�rten}}. 

A partir de entonces se dedic� a componer fragmentos muy breves cuya estructura era definida por {\bf motivos} y no por la armon�a. Era esto lo que sol�a ocurrir en formas musicales anteriores como la forma sonata. A este periodo en sus composiciones se le llama atonalidad libre, aunque cabe destacar que Schoenberg rechazaba fervientemente este t�rmino:

\begin{quote}
\emph{La expresi�n ``m�sica atonal'' es de lo m�s desafortunada \textemdash es como llamar a volar ``el arte de no caer'' o a nadar ``el arte de no ahogarse''.}\\
{\footnotesize Mencionado en \cite{hauer} A. Schoenberg, \emph{Hauer's Theories}, en \emph{Style and Idea}, 1923.}
\end{quote}

\noindent A este periodo pertenece tambi�n su famoso ciclo de canciones \href{https://www.youtube.com/watch?v=vQVkbKULKpI}{\emph{Pierrot Lunaire}}, Op. 21 (1912). Su nombre completo es \textit{Tres veces siete poemas de Pierrot Lunaire de Albert Giraud}, ya que est� dividida en 3 grupos de 7 canciones cada uno, cuyos textos son una selecci�n de 21 poemas del ciclo hom�nimo de Albert Giraud. 

Se encuentran en ella abundantes referencias al n�mero 7. Schoenberg hace un uso extensivo de motivos de 7 notas a lo largo de la obra, mientras que el conjunto musical que la interpreta, incluyendo al director, consta de 7 miembros. De hecho, a este conjunto de instrumentos \textemdash flauta, clarinete, viol�n, violonchelo, piano y voz\textemdash~se le ha dado el nombre de \textit{ensemble Pierrot} en su honor.  Otros n�meros importantes en la obra son el 3 y el 13. Cada poema consta de 13 l�neas, mientras que la primera l�nea de cada poema aparece 3 veces, en las l�neas 1, 7 y 13.

En esta obra no s�lo hay una ausencia total de relaciones tonales, sino que el tratamiento vocal evita tambi�n cualquier relaci�n est�tica con las t�cnicas tradicionales: es un \emph{Sprechgesang}, un canto hablado. De hecho, Schoenberg se refiere a estas piezas no como canciones, sino como melodramas. V�ase~\cite{diaz} para m�s informaci�n.

\subsection{El surgimiento del sistema dodecaf�nico}
Schoenberg no estaba a�n satisfecho con su t�cnica compositiva, ya que admiraba las obras extensas de los m�sicos rom�nticos y pensaba que su atonalidad no pod�a sostener una obra de gran envergadura. Es decir, necesitaba un hilo conductor mejor que los motivos para poder componer obras atonales m�s largas.

Por aquella �poca sufri� crisis en varios aspectos de su vida. En lo personal, su mujer Matilde Zemlinsky acababa de abandonarlo por otro hombre, aunque posteriormente volver�a junto al compositor. Y en lo profesional, sus obras no eran del gusto del p�blico, por lo que no contaba con suficiente dinero para mantener a su familia. Todas estas circunstancias, unidas al desarrollo de la Primera Guerra Mundial, no le permitieron componer apenas entre 1914 y 1923.

Tras el final de la guerra, en 1919, Schoenberg fund� la Sociedad para Interpretaciones Musicales Privadas junto a sus disc�pulos y amigos Alban Berg y Anton Webern. Schoenberg, Berg y Webern se autodenominaron la Segunda Escuela de Viena en honor al grupo de compositores del siglo XVIII Haydn, Mozart y Beethoven, quienes formaban la Primera Escuela de Viena.

En la Sociedad para Interpretaciones Musicales Privadas se presentaba m�sica contempor�nea en circunstancias que favorecieran su adecuada apreciaci�n. As� se evitaba que dichas obras, al no ser entendidas por el p�blico, fueran inmediatamente rechazadas. Las obras de compositores como Mahler, Debussy, Bart�k, Ravel, Strauss y Stravinsky se incluyeron en los programas de conciertos organizados por la Sociedad.

En este contexto Schoenberg pudo reflexionar sobre sus t�cnicas compositivas, y al fin public� en 1923 su ensayo \emph{M�todo de composici�n con doce sonidos}~\cite{scho}, donde se describ�an por primera vez los axiomas del dodecafonismo. Estos axiomas constitu�an la soluci�n al problema de la atonalidad libre que tanto le hab�a estado atormentando durante una d�cada.

Su primera obra �ntegramente dodecaf�nica, publicada tambi�n en 1923, es la Suite para piano Op. 25, que podr�n ver a continuaci�n. Es la pieza m�s temprana en la que Schoenberg usa series dodecaf�nicas en cada uno de los movimientos. En dos obras anteriores a ella usa series dodecaf�nicas, pero en movimientos aislados: la Op. 23, \href{https://www.youtube.com/watch?v=7A9HSlgDlQE}{\emph{5 St\"ucke}} (1920\textemdash23), en el movimiento de Waltz final; y su \href{https://www.youtube.com/watch?v=fzAFalLbXxg}{Serenata}, Op. 24, en su Soneto central.

Las series utilizadas en la Suite Op. 25 servir�n de ejemplo en este texto, y su tercer movimiento, \href{https://www.youtube.com/watch?v=scwNtGdop6w}{Musette}, ser� estudiado y analizado en el apartado \ref{musette} con el fin de entender una obra dodecaf�nica en toda su extensi�n. A continuaci�n el lector podr� escuchar la Suite para piano Op. 25:

\url{https://www.youtube.com/watch?v=39x0Ypi4gTc}%
	\section[EL SISTEMA DODECAF\'ONICO DE SCHOENBERG]{EL SISTEMA DODECAF\'ONICO DE SCHOENBERG}\label{ch:dodecafonismo}
	\subsection{Los postulados del dodecafonismo}
		El dodecafonismo es un sistema compositivo que predetermina la melod\'ia y la armon\'ia a partir de una ordenaci\'on de las doce notas de la escala crom\'atica, que se llama \textit{serie}. \'Esta y algunas de sus transformaciones son los ladrillos con los que se construyen las alturas de las notas; son el \'unico material que se puede utilizar. 
		
		El resto de elementos de la pieza, como el n\'umero de instrumentos, el ritmo, el car\'acter, la textura o las din\'amicas, se dejan a discreci\'on del compositor. No serializar todos los conjuntos ser\'a la principal cr\'itica al dodecafonismo por parte de los compositores serialistas que sucedieron a su creador, Arnold Schoenberg. Para los serialistas integrales, como Pierre Boulez, aquello restaba cohesi\'on al modelo compositivo; para los dodecafonistas, aportaba libertad. \cite{boulez}
		
		Precisamente la predeterminaci\'on dodecaf\'onica, aunque parece limitante, permite realizaciones musicales y estilos de composici\'on muy diferentes: Schoenberg daba un tratamiento tradicional a sus obras, ya que a\'un admiraba las formas cl\'asicas; Alban Berg iba m\'as all\'a al utilizar series que recordaban a las tr\'iadas tonales; y, en cambio, Anton Webern evitaba radicalmente cualquier asociaci\'on con la tradici\'on. 
		
		Schoenberg defini\'o su sistema musical a partir de cuatro postulados que, en realidad, se basan en principios matem\'aticos \cite{dominguez}:
		
		\emph{1. La serie \emph{[sobre la que se construye la obra dodecaf\'onica]} consta de las doce notas de la escala crom\'atica dispuestas en un orden lineal espec\'ifico.}
		
		\emph{2. Ninguna nota aparece m\'as de una vez en la serie.}
		
		Los dos primeros postulados expresan que una obra dodecaf\'onica fundamenta su estructura sobre una permutaci\'on de la escala de doce semitonos. Dicha permutaci\'on $\sigma$ es una biyecci\'on del conjunto numerado de las doce notas \{Do = 0, Do\# = 1, Re = 2, Re\# = 3, Mi = 4, Fa = 5, F\# = 6, Sol = 7, Sol\# = 8, La = 9, La\# = 10, Si = 11\} consigo mismo, y se representa de esta forma:
		
		\[
			\drow{
				\sigma(0),\sigma(1),\sigma(2),\sigma(3),\sigma(4),\sigma(5),\sigma(6),\sigma(7),\sigma(8),\sigma(9),\sigma(10),\sigma(11)
			}
		\]
	
		La permutaci\'on $\sigma(m)$, con $m\in \mathbb{Z} / (12)$\footnote{$\mathbb{Z} / (12)=\{0,\ 1,\ 2,\ 3,\ 4,\ 5,\ 6,\ 7,\ 8,\ 9,\ 10,\ 11\}$}, pertenece al grupo sim\'etrico de orden 12: $\sigma\in$ S$_{12}$. Por ejemplo, en la Suite para piano Op. 25 Schoenberg utiliza como serie original en todos los movimientos de la obra la siguiente permutaci\'on $\sigma$:
		
		\[\sigma=\drow{4,5,7,1,6,3,8,2,11,0,9,10}\]	
		\begin{center}
			\includegraphics[width=12.1cm]{1.png}
		\end{center}
		
		\emph{3. La serie ser\'a expuesta en cualquiera de sus aspectos lineales: original, inversi\'on, retrogradaci\'on de la original y retrogradaci\'on de la inversi\'on.}
		 
		\emph{4. La serie puede usarse en sus cuatro aspectos desde cualquier nota de la escala.}
		
		Los dos \'ultimos postulados ampl\'ian los recursos compositivos al admitir la transformaci\'on de la serie original mediante \emph{inversi\'on}, \emph{retrogradaci\'on}, \emph{inversi\'on retr\'ograda} y \emph{transposici\'on}\footnote{No confundir con un 2-ciclo. Una transposici\'on musical se corresponde con una traslaci\'on matem\'atica.}. El compositor puede utilizar cualquiera de las transformaciones de una serie al componer su obra dodecaf\'onica. El conjunto de series que puede utilizar, que viene dado por la serie original y todas sus posibles transformaciones, se conoce como \emph{espectro serial}. \cite{dominguez}
		
	\subsection{Las transformaciones de una serie}
		\label{transPsi}
		Transformar una serie es matem\'aticamente equivalente a aplicar una funci\'on sobre la serie, y que asocie esa permutaci\'on a la permutaci\'on transformada. Por tanto, cualquier funci\'on transformativa $\Psi$ se aplica sobre el conjunto de las permutaciones, S$_{12}$.
		
	\subsubsection{Transposiciones}
		La \emph{transposici\'on}, mencionada en el cuarto postulado, consiste en subir o bajar la serie original un n\'umero determinado de semitonos. Por tanto, no se modifican los intervalos entre las notas, sino solamente la altura a la que est\'a la serie. Ya que consideraremos todas las octavas equivalentes, debemos trabajar m\'odulo 12. 
		
		La serie transportada k semitonos (con k constante),
		
		$\mbox{T}^{{\mbox{k}}}(\sigma)$, se construye sumando k a $\sigma$ (mod. 12):
		\[\mbox{T}^{{\mbox{k}}}(\sigma(m))=\sigma(m)+{\mbox{k}}\]		
		{\[
		\mbox{T}^{{\mbox{k}}}=
		\left(\begin{array}{*{7}c}
			0&1&2&&9&10&11\\
			\sigma(0)+{\mbox{k}}&\sigma(1)+{\mbox{k}}&\sigma(2)+{\mbox{k}}&
			\cdots&
			\sigma(9)+{\mbox{k}}&\sigma(10)+{\mbox{k}}&\sigma(11)+{\mbox{k}}\\
		\end{array}\right)
		\]}
		
		A su vez, T$^{\mbox{k}}$ se forma al componer k transposiciones de 1 semitono: $\mbox{T}^{\mbox{k}}=\mbox{T}^1\circ\mbox{T}^1\circ\ldots\circ\mbox{T}^1$, k veces. Debido a que k es en realidad el exponente en la potencia de T, se coloca este n\'umero como super\'indice.
		
		Hist\'oricamente, la notaci\'on $\Psi_{\mbox{k}}$, $\Psi^{\mbox{k}}$ o $\Psi({\mbox{k}})$ se ha usado en sustituci\'on de la composici\'on de la transposici\'on T$^{\mbox{k}}$ y otra funci\'on $\Psi$, en el respectivo orden: $\Psi^{\mbox{k}}=\Psi \circ \mbox{T}^{\mbox{k}} = \Psi(\mbox{T}^{\mbox{k}})$. Sin embargo, esta notaci\'on es especialmente ambigua y confusa, sobre todo al trabajar con funciones no conmutativas. Por ello, es preferible ce\~nirse a la notaci\'on estrictamente matem\'atica; es decir, a la composici\'on de funciones, aun omitiendo $\circ$: \cancel{$\Psi_{\mbox{k}}$}, \cancel{$\Psi^{\mbox{k}}$}, \cancel{$\Psi({\mbox{k}})$} $\rightarrow \Psi\mbox{T}^{\mbox{k}}$
		
		Una posible serie transportada sobre la permutaci\'on $\sigma$ de la Suite para piano Op. 25, con k $= 6$, es la siguiente serie T$^6$:
		\[\mbox{T}^6=\drow{10,11,1,7,0,9,2,8,5,6,3,4}\]	
		\begin{center}
			\includegraphics[width=12.1cm]{2.png}
		\end{center}
		
	\subsubsection{Retrogradaci\'on}
		La \emph{retrogradaci\'on} consiste en leer la serie original desde la nota final hacia atr\'as, es decir, aplicar a la serie una simetr\'ia especular. De este modo, la primera nota ir\'a al \'ultimo puesto, la segunda al pen\'ultimo, y as\'i sucesivamente.
		
		La serie retr\'ograda se construye de esta forma:
		\[\mbox{R}(\sigma(m))=\sigma(-1-m)\]
		\[{\mbox{R}=\drow{\sigma(11),\sigma(10),\sigma(9),\sigma(8),\sigma(7),\sigma(6),\sigma(5),\sigma(4),\sigma(3),\sigma(2),\sigma(1),\sigma(0)}}\]
			
		La serie retr\'ograda sobre la permutaci\'on $\sigma$ de la Suite Op. 25 es la siguiente serie R:	
		\[\mbox{R}=\drow{10,9,0,11,2,8,3,6,1,7,5,4}\]		
		\begin{center}
			\includegraphics[width=12.1cm]{3.png}
		\end{center}
		
	\subsubsection{Inversi\'on}
		La \emph{inversi\'on} consiste en cambiar la direcci\'on --de ascendente a descendente, y viceversa-- de los intervalos entre cada nota de la serie. Si el primer intervalo en la serie original $\sigma$ es de $+k$, el primer intervalo en la serie invertida I ser\'a de $-k$ (mod. 12), por lo que debemos cambiar el signo de $\sigma$ para construir I. Adem\'as, queremos que la primera nota de ambas series, I(0) y $\sigma$(0), coincidan, as\'i que debemos transportar la serie ($-\sigma$) un n\'umero $\lambda$ de semitonos para que esta condici\'on se cumpla:
		\begin{align*}
		\mbox{I}(0)=-\sigma(0)+\lambda&=\sigma(0)\\
		\Longrightarrow \lambda&=2\sigma(0)
		\end{align*}
		Por tanto, la serie invertida se construye de esta forma:
		\[\mbox{I}(\sigma(m))=-\sigma(m)+2\sigma(0)\]
		
		\[\mbox{I}=\left(\begin{matrix}0&1&2&&10&11\\\sigma(0)&-\sigma(1)+2\sigma(0)&-\sigma(2)+2\sigma(0)&\ldots&-\sigma(10)+2\sigma(0)&-\sigma(11)+2\sigma(0)\\\end{matrix}\right)\]
		
		La serie invertida sobre la permutaci\'on $\sigma$ de la Suite Op. 25 es la siguiente serie I:
		\[\mbox{I}=\drow{4,3,1,7,2,5,0,6,9,8,11,10}\]		
		\begin{center}
			\includegraphics[width=12.1cm]{4.png}
		\end{center}
				
		En total, obtendremos 48 series -- aunque no obligatoriamente distintas entre s\'i -- pertenecientes a un solo espectro serial. Hay 12 series originales sobre cada una de las doce notas, 12 series retr\'ogradas, 12 invertidas y 12 series sobre las que se aplica tanto la retrogradaci\'on como la inversi\'on. A continuaci\'on se muestra la sintaxis simple junto a la matem\'atica:
		
		\begin{multicols}{2}
			\underline{Sintaxis simple}
			
			T$_0$, T$_1$, T$_2$\ldots
			
			R$_0$, R$_1$, R$_2$\ldots
			
			I$_0$, I$_1$, I$_2$\ldots
			
			IR$_0$, IR$_1$, IR$_2$\ldots
			
			\underline{Sintaxis matem\'atica}
			
			T$^0$, T$^1$, T$^2$\ldots
			
			R, RT$^1$, RT$^2$\ldots
			
			I, IT$^1$, IT$^2$\ldots
			
			IR, IRT$^1$, IRT$^2$\ldots
		\end{multicols}
	
	\subsection{Matrices dodecaf\'onicas}
		
		Dada una serie, su matriz dodecaf\'onica es una representaci\'on visual de su espectro serial; es decir, del conjunto de series derivadas de esa serie. El espectro serial es todo el material compositivo sonoro del que se dispone para la composici\'on de una obra dodecaf\'onica. Al poder ordenar y disponer la informaci\'on en una tabla, el compositor puede acceder a toda ella al mismo tiempo sin tener que calcular cada serie individualmente.		
		
		La matriz se lee en la direcci\'on en la que aparece el nombre de la serie. Las series T se leen de izquierda a derecha, mientras que las series R de derecha a izquierda. Las series I se leen de arriba a abajo y las IR/RI de abajo a arriba.
		
		He creado un programa que devuelve en formato \LaTeX{} la matriz correspondiente a cualquier serie dodecaf\'onica que se introduzca en teclado, adem\'as de producir la nomenclatura simple para cada serie. El c\'odigo, escrito en C++, %est\'a incluido en el Anexo \ref{app:matrices}, p\'agina \pageref{app:matrices}.
		se puede encontrar en el enlace \url{https://gitlab.com/dodecafonismo/cppmatrices}.
		
		A continuaci\'on, se incluye la matriz dodecaf\'onica de la serie P de la Suite Op. 25 de Schoenberg. Mientras que la mayor\'ia de tablas tienen dos filas inferiores, que se corresponden con las distintas nomenclaturas de RI e IR para una misma serie -- ya que normalmente no conmutan --, en la matriz de la serie P s\'i coinciden%, como se mencionar\'a en el apartado \ref{conmut}
		.
		
		{$\begin{array}{l|cccccccccccc|r}
		&\mbox{I}_{0}&\mbox{I}_{1}&\mbox{I}_{3}&\mbox{I}_{9}&\mbox{I}_{2}&\mbox{I}_{11}&\mbox{I}_{4}&\mbox{I}_{10}&\mbox{I}_{7}&\mbox{I}_{8}&\mbox{I}_{5}&\mbox{I}_{6}&\\
		\hline
		\mbox{T}_{0}&4&5&7&1&6&3&8&2&11&0&9&10&\mbox{R}_{0}\\
		\mbox{T}_{11}&3&4&6&0&5&2&7&1&10&11&8&9&\mbox{R}_{11}\\
		\mbox{T}_{9}&1&2&4&10&3&0&5&11&8&9&6&7&\mbox{R}_{9}\\
		\mbox{T}_{3}&7&8&10&4&9&6&11&5&2&3&0&1&\mbox{R}_{3}\\
		\mbox{T}_{10}&2&3&5&11&4&1&6&0&9&10&7&8&\mbox{R}_{10}\\
		\mbox{T}_{1}&5&6&8&2&7&4&9&3&0&1&10&11&\mbox{R}_{1}\\
		\mbox{T}_{8}&0&1&3&9&2&11&4&10&7&8&5&6&\mbox{R}_{8}\\
		\mbox{T}_{2}&6&7&9&3&8&5&10&4&1&2&11&0&\mbox{R}_{2}\\
		\mbox{T}_{5}&9&10&0&6&11&8&1&7&4&5&2&3&\mbox{R}_{5}\\
		\mbox{T}_{4}&8&9&11&5&10&7&0&6&3&4&1&2&\mbox{R}_{4}\\
		\mbox{T}_{7}&11&0&2&8&1&10&3&9&6&7&4&5&\mbox{R}_{7}\\
		\mbox{T}_{6}&10&11&1&7&0&9&2&8&5&6&3&4&\mbox{R}_{6}\\
		\hline
		&\mbox{IR}_{0}&\mbox{IR}_{1}&\mbox{IR}_{3}&\mbox{IR}_{9}&\mbox{IR}_{2}&\mbox{IR}_{11}&\mbox{IR}_{4}&\mbox{IR}_{10}&\mbox{IR}_{7}&\mbox{IR}_{8}&\mbox{IR}_{5}&\mbox{IR}_{6}&\\
		\hline
		&\mbox{RI}_{0}&\mbox{RI}_{1}&\mbox{RI}_{3}&\mbox{RI}_{9}&\mbox{RI}_{2}&\mbox{RI}_{11}&\mbox{RI}_{4}&\mbox{RI}_{10}&\mbox{RI}_{7}&\mbox{RI}_{8}&\mbox{RI}_{5}&\mbox{RI}_{6}&
		\end{array}$}
	
		Por otro lado, he escrito otro programa en el propio lenguaje \LaTeX{} que crea esta misma tabla con el comando \verb|\dmatrix|, y tiene cualquier serie como argumento: \verb|\dmatrix{4,5,7,1,6,3,8,2,11,0,9,10}|. %El c\'odigo se encuentra en el Anexo \ref{app:latex}, p\'agina \pageref{app:latex1}. 
		El c\'odigo se encuentra en el paquete de \LaTeX{} \texttt{ddphonism}, incluido en el enlace  \url{https://gitlab.com/dodecafonismo/ddphonism}.
		La tabla aparece sin el orlado de nomenclaturas:
		\dmatrix{4,5,7,1,6,3,8,2,11,0,9,10}
		
		Tambi\'en he creado una p\'agina interactiva que genera matrices de cualquier serie para cualquier longitud serial, adem\'as de generar series aleatorias. Permite escoger entre dos numeraciones y dos nomenclaturas. Est\'a escrita en Elm y el c\'odigo puede encontrarse en \url{https://gitlab.com/dodecafonismo/matrices}.
		
%			\qrcode{https://matrices.netlify.com/}
			
			En este enlace se encuentra la aplicaci\'on web. Sus instrucciones de uso se encuentran al final de la p\'agina: \url{https://matrices.netlify.com/}.%
	\chapter{ANÁLISIS DE UNA OBRA DODECAFÓNICA: OP. 25}
\label{suitechapter}
	\section{Series de la Suite op. 25}
		Lo primero que hará un compositor dodecafónico antes de empezar a componer será escoger su serie original. Su elección nunca es una simple cuestión de azar; al contrario, ya que las singularidades de la serie darán un carácter especial a toda la obra. Por ejemplo, el compositor puede escoger una serie con simetrías, y así tendrá series repetidas entre su espectro serial. También puede tener simetrías internas solo en un fragmento de tres o cuatro notas, y de este modo podrá el compositor oscilar entre varias series del espectro que se parezcan entre sí.\footnote{Para un estudio completo de las relaciones de similitud entre series se recomienda \emph{On the Similarity of Twelve-Tone Rows}, de Tuukka Ilomäki. \cite{ilomaki}}
		
		En la Suite para Piano Op. 25, Schoenberg escoge su serie $\sigma$ para resaltar el intervalo de tritono (6 semitonos). A continuación se observan en negrita los intervalos entre las notas de esta serie, en unidad de semitono:
		
		\vspace{-0.5cm}
		\begin{footnotesize}	$$\left(\begin{matrix}0&&1&&2&&3&&4&&5&&6&&7&&8&&9&&10&&11&\\4&\mathbf{1}&5&\mathbf{2}&7&\mathbf{6}&1&\mathbf{5}&6&\mathbf{9}&3&\mathbf{5}&8&\mathbf{6}&2&\mathbf{9}&11&\mathbf{1}&0&\mathbf{9}&9&\mathbf{1}&10&\mathbf{6}\end{matrix}\right)$$		
		\end{footnotesize}\vspace{-0.5cm}
				
		Presenta repeticiones triples de los intervalos de tritono (6), de sexta mayor (9) y de segunda menor o semitono (1): los intervalos más disonantes; una repetición doble de cuarta justa (5), y un intervalo de segunda mayor (2); además de una consecución de intervalos repetida: 9 -- 1 -- 9 -- 1. Como se forma el intervalo de tritono al enlazar la serie original con una serie que empiece por la misma nota, se tiene en cuenta el intervalo de tritono (6) al final. En el dodecafonismo se evitan deliberadamente los intervalos de tercera mayor (4), ya que estos son la base de la eludida armonía tonal. \label{serie25}
		
		El intervalo de tritono tiene la particularidad de no modificarse en la inversión y transportación k = 6, por lo que estos intervalos aparecen en los lugares originales, mientras que en los procedimientos de retrogradación y retrogradación inversa ocupan sus lugares en retrógrado. En particular, Schoenberg utiliza entre los seis movimientos de la Suite solamente las ocho series de todo el espectro serial que cumplen estos requisitos: T$^0$, T$^6$, I, IT$^6$, R, RT$^6$, RI y RIT$^6$, que podemos observar en el Anexo \ref{app:series}, página \pageref{app:series}.
		
		Estas series tienen muchos elementos en común: todas comienzan o acaban por Mi$\natural$ o por Si$\flat$, lo que permite enlazar unas series con otras por medio del unísono o del tritono; se mantienen los intervalos de tritono en sus lugares originales o retrógrados, y coinciden en las dos primeras y las dos últimas notas dos a dos.
		
		Se han realizado estudios -- como el de Martha Hyde \cite{hyde} -- en los que se limitan las series utilizadas en la Suite a cuatro: T$^0$, T$^6$, I e IT$^6$, pero ya que el objetivo de este texto no es analizar la obra entera se dejará esta cuestión para análisis posteriores.
		
	\section{Descripción de la Suite op. 25}
		Schoenberg realiza en la serie $\sigma$ una partición triple; es decir, la serie se divide en tres tetracordios, y cada uno de ellos contiene un intervalo de tritono. El último tetracordio, si se retrograda, consta de las notas 10 -- 9 -- 0 -- 11, que en notación germánica es la secuencia BACH. Esto puede ser un homenaje al compositor Johann Sebastian Bach (1685—1750), ya que Schoenberg admiraba a los grandes compositores anteriores a él por las estructuras formales de sus obras. \cite{xiao}
		
		Otro posible homenaje a Bach y sus contemporáneos barrocos es precisamente la forma de la obra: es una Suite, género cultivado durante los siglos XVII y XVIII que se compone de una variedad de danzas. La Suite de Schoenberg está formada por seis danzas: un Preludio, una Gavota, una Musette, un Intermezzo -- que no tiene influencia barroca sino más bien de Brahms, otro modelo para Schoenberg --, un Minueto con Trío y una Giga. Además, el estilo, la textura -- contrapuntística, típicamente barroca -- y la estructura de cada danza se corresponden con los estilos, texturas y estructuras de las danzas homónimas del periodo bachiano.
        
        Por ser ésta su primera obra totalmente dodecafónica, Schoenberg la utilizó como una muestra al mundo de las posibilidades de su nuevo método compositivo. Fue también por lo que tomó un formato tan variado como una Suite: así podía en una misma obra componer con estilos tan distintos como los de las distintas danzas.
        
        Al componer la obra, Schoenberg trata cada tetracordio como una subunidad individual. Los superpone contra otras series del espectro también divididas, o utiliza sus notas como un solo acorde cuatríada. Estas divisiones no sólo sirven para hacer la serie más reconocible o añadir cohesión a la obra, sino que además facilitan el desarrollo de la serie específicamente en el estilo de cada danza.
		
	\section{Análisis de la Musette}
		En el tercer movimiento de la Suite, la Musette, Schoenberg recrea la danza barroca que toma su nombre del instrumento homónimo: la \emph{cornamusa}, de la familia de la gaita. La música compuesta para estos instrumentos suele consistir en una melodía acompañada por una nota pedal, que se traduce aquí en la presencia de un bordón sobre el Sol$\natural$ (nota 7). Esta nota se extrae de cada una de las series utilizadas y se forma con ella un ostinato rítmico en la mano izquierda del piano. Con el resto de sonidos de cada serie, Schoenberg vuelve a emular el estilo de la danza barroca y articula un discurso polifónico a dos voces con ritmos esencialmente cortos.
		
		A partir de la doble barra del compás 9, el Re$\flat$ (nota 1) acompaña a Sol$\natural$ y ambos crean un doble bordón en la mano izquierda. La elección de esas dos notas está estrechamente relacionada con la tradicional relación de quinta justa formada por Sol$\natural$ y Re$\natural$ en la música tonal. Schoenberg sustituye las quintas justas tonales por los tritonos dodecafónicos, subrayando aún más su <<emancipación de la disonancia>>.
		
		Además de las similitudes texturales, rítmicas y armónicas, la Musette de Schoenberg comparte estructura formal con las danzas barrocas. Y esta semejanza es quizás la más notable, ya que fue la búsqueda de estructura formal lo que inspiró a Schoenberg a desarrollar su método compositivo. La Musette barroca, como todos los movimientos de danza, presenta una estructura binaria con simetría tonal: empieza y acaba por la misma tonalidad, mientras que el centro es zona de desarrollo. Schoenberg despoja de funcionalidad tonal a esa simetría, madre de la forma sonata, y la aplica a su composición dodecafónica.
		
		En este movimiento se pueden diferenciar a simple vista tres secciones, divididas en los compases 9 y 20, debido a cambios de textura, figuración y tempo. En la segunda sección se le añade melodía a la mano izquierda del piano, dejando más camuflado el bordón que en la primera sección, además de que éste se vuelve doble, mientras que vuelve a aparecer claramente en la tercera sección. También en la segunda sección aparece una nueva figuración, que es la semicorchea; y, por último, en los dos compases de división aparecen dos \emph{a tempo}, que marcan el final de las dos primeras secciones tras dos zonas de variabilidad rítmica. \cite{clercq}
		
		Para que esta estructura tríptica sea una forma binaria, la primera y la última parte deben mantener un parecido, que se observa a través del análisis de las series utilizadas en el movimiento. Estas series son T$^0$, T$^6$, I e IT$^6$.
		
		En la Musette, Schoenberg hace un uso casi absoluto de la tripartición serial, hasta el punto de individualizar los tetracordios por separado y concederles privilegios seriales, como la retrogradación. Por ejemplo, en el compás 7, en la voz inferior de la mano derecha aparece el tetracordio 4 -- 5 -- 2 -- 3, que es o bien el primer tetracordio de RIT$^6$ o la retrogradación del tercer tetracordio de IT$^6$, mientras que los otros dos tetracordios de IT$^6$, 10 -- 9 -- 7\footnote{La nota 7 aparece como bordón y no en la misma voz que el resto del tetracordio, por lo que su posición es también excepcional.} -- 1 en la voz superior y 8 -- 11 -- 6 -- 0 en la mano izquierda, aparecen en el orden correcto. Entonces no se puede analizar el compás como RIT$^6$, sino indicar que hay una alteración puntual de IT$^6$.
		
		Por tanto, es muy complicado analizar esta obra en su totalidad, ya que la flexibilidad en la ordenación de los tetracordios puede generar situaciones muy ambiguas. Debido a estas fragmentaciones y a las variadas combinaciones de tetracordios originales y retrógrados, se escucha un área de desarrollo hacia la sección media del movimiento. En cambio, las series al principio y al final de la pieza se presentan casi íntegramente, como una exposición y reexposición. He aquí un vínculo con la simetría de las formas binarias tonales. \cite{clercq}
		
		Es más, incluso el orden de las series utilizadas en la primera y en la última sección coinciden, exceptuando dos repeticiones consecutivas y las series T$^0$ finales, que actúan como una cadencia serial:
		$$\begin{matrix}\mathbf{c}.\mathbf{1}&\text{T}^0&\text{IT}^6&\text{T}^6&\text{I}&\text{T}^0&\text{I}&\text{T}^6&\begin{matrix}\text{IT}^6&\text{IT}^6\\\end{matrix}&&&\mathbf{c}.\mathbf{9}\\\mathbf{c}.\mathbf{22}&\text{T}^0&\text{IT}^6&\text{T}^6&\text{I}&\text{T}^0&\begin{matrix}\text{I}&\text{I}\\\end{matrix}&\text{T}^6&\text{IT}^6&\text{T}^0&\text{T}^0&\mathbf{c}.\mathbf{31}\\\end{matrix}$$
		
		En el Anexo \ref{app:score}, página \pageref{app:score}, se encuentra el análisis serial completo de la Musette, y en la pista 1 su reproducción con el programa Musescore.%
	\section{DEFINICIONES MATEMÁTICAS}\label{ch:permutaciones}
	\subsection{Conjuntos y grupos}
		Un \emph{conjunto} es una colección de objetos bien definidos y distintos entre sí que se llaman \emph{elementos}. 
	
		Para definir un conjunto se puede o bien listar los objetos uno a uno, o bien describirlos por medio de un predicado: una o varias propiedades que caracterizan a todos los elementos de dicho conjunto.

		Por ejemplo, el conjunto K$_{\mbox{i}}$, formado por las doce notas de la escala cromática de una misma octava i, está bien definido porque podemos hacer una lista con ellas: por ejemplo, $\mbox{K}_{\mbox{4}} = $
		
		{$\{\mbox{Do}_{\mbox{4}}, \mbox{Do\#}_{\mbox{4}}, \mbox{Re}_{\mbox{4}}, \mbox{Re\#}_{\mbox{4}}, \mbox{Mi}_{\mbox{4}}, \mbox{Fa}_{\mbox{4}}, \mbox{Fa\#}_{\mbox{4}}, \mbox{Sol}_{\mbox{4}}, \mbox{Sol\#}_{\mbox{4}}, \mbox{La}_{\mbox{4}}, \mbox{La\#}_{\mbox{4}}, \mbox{Si}_{\mbox{4}}\}$}
		
		Por un lado, aun llamando a las notas de distinta manera, el conjunto, conceptualmente, es el mismo. Además, el hecho de listar algún elemento más de una vez no afecta a su definición. Como $\mbox{Do\#}_{\mbox{4}} = \mbox{Re}\flat_{\mbox{4}}$ ({ya que trabajamos con temperamento igual}), $\mbox{K}_{\mbox{4}}$ también puede ser listado así:
		
		{$\{\mbox{Do}_{\mbox{4}}, \mbox{Do\#}_{\mbox{4}}, \mbox{Re}\flat_{\mbox{4}}, \mbox{Re}_{\mbox{4}}, \mbox{Re\#}_{\mbox{4}}, \mbox{Mi}_{\mbox{4}}, \mbox{Fa}_{\mbox{4}}, \mbox{Fa\#}_{\mbox{4}}, \mbox{Sol}_{\mbox{4}}, \mbox{Sol\#}_{\mbox{4}}, \mbox{La}_{\mbox{4}}, \mbox{La\#}_{\mbox{4}}, \mbox{Si}_{\mbox{4}}\}$}
	
		En cambio, el conjunto D, formado por las duraciones rítmicas elementales -- sin ligaduras ni puntillos --, es infinito, por lo que no se puede listar de forma completa. Sin embargo, se puede expresar por medio de un predicado:		
		\[\mbox{D} =\{2^n:n\in\mathbb{Z},\ n\le 2\} = \{4,\ 2,\ 1,\ \dfrac{1}{2},\ \dfrac{1}{4},\ \dfrac{1}{8},\ \ldots\} = \{\Ganz,\ \Halb,\ \Vier,\ \textsf{\Acht},\ \ldots\}\]
		
		La notación $n\in\mathbb{Z}$ significa que $n$ pertenece a los números enteros. En este caso se han representado las duraciones mediante su ratio con la duración de la negra.
	
		Los elementos de un conjunto pueden combinarse mediante \emph{operaciones} -- como la suma o la multiplicación en el caso de los números -- para dar otros objetos matemáticos. 
		
		Se dice que un conjunto G no vacío y una operación binaria ($\ast$) forman la estructura de un \emph{grupo} (G, $\ast$) cuando cumplen:
	
		\begin{enumerate}
			\item{Su operación es interna: Si $a,b\in$ G, entonces $a\ast b\in$ G.}		
			\item{Su operación es asociativa: Si $a,b,c\in$ G, $(a\ast b)\ast c=a\ast(b\ast c)$. }
			\item{Existe un elemento $e$ en G, llamado elemento neutro o identidad, tal que para todo $x\in$ G se cumple que $e\ast x = x\ast e = x$. Se puede probar que el neutro es único para cada grupo. A veces se incluye dentro de la definición del grupo: (G, $\ast$, $e$).}
			\item{Cada $x\in$ G tiene asociado otro elemento $x^{-1}\in$ G, llamado elemento inverso, tal que $x \ast x^{-1} = x^{-1}  \ast x = e$. Se puede probar que el inverso de cada elemento es único.}		
		\end{enumerate}
	
	($\mathbb{Z},+,0$) y ($\mathbb{Q},+,0$) son grupos, pero ($\mathbb{N},+,0$) no porque no existe el \textit{inverso} de 2 con la suma: $-2\notin\mathbb{N}$. ($\mathbb{R},*,1$) y ($\mathbb{Q},*,1$) son grupos, pero ($\mathbb{Z},*,1$) no porque no existe el \textit{inverso} de 2 con la multiplicación: $\dfrac12\notin\mathbb{Z}$.
	
	\subsection{Funciones y permutaciones}
		Una \emph{función} es una regla que asocia a cada elemento de un primer conjunto, llamado \emph{dominio}, un único elemento de un segundo conjunto. Si la función se llama $f$, el dominio A y el segundo conjunto B, se denota $f:\mbox{A}\to \mbox{B}$. El elemento asociado a un $x$ mediante $f$ se denota $f(x)$.
		
		Todos los $x\in$ A tienen que estar asociados a un $f(x)\in$ B, pero no todos los elementos de B tienen un elemento de A asociado. Los elementos de B que sí lo cumplen, es decir, los que se pueden escribir como $f(x)$ para algún $x$, forman el conjunto \emph{imagen} de la función: $im(f)=\{\ y\in \mbox{B}:\ \exists\ x \in \mbox{A},\ f(x)=y\ \}$
		
		Cuando varias funciones se aplican una detrás de la otra decimos que realizamos la operación de \textit{composición de funciones}. Se representa con el símbolo $\circ$. La imagen de la primera función será el dominio de la segunda, y así sucesivamente. Por ejemplo, aplicar una función $f(x)$ y después aplicar una función $g(x)$ se denota $g(f(x))=(g\circ f)(x)$.
		
		Una \emph{permutación} $\sigma$(X) es una función sobre un conjunto X que asocia sus elementos a los elementos del mismo conjunto X de manera unívoca. Es decir, asocia cada elemento a uno, y solo uno, de los elementos de su mismo conjunto ($\sigma:\mbox{X}\to \mbox{X}$).

		El conjunto de todas las posibles permutaciones sobre un determinado conjunto X, junto con la operación de composición de funciones ($\circ$), forma un grupo denotado por S$_{\mbox{x}}$. Para probarlo, se debe comprobar que cumple todas las propiedades de los grupos.

		\begin{enumerate}
			\item{Permutar dos veces es también una permutación.}
			\item{La composición de funciones es asociativa.}
			\item{La permutación que asigna un elemento a sí mismo es la función identidad.}
			\item{Como las permutaciones son biyectivas, cada una tiene una inversa que es también una permutación.}		
		\end{enumerate}

		Cuando X es el conjunto de números naturales desde 1 hasta $n$, el grupo S$_{\mbox{x}}$ se representa como S$_n$ y se le denomina el grupo simétrico de orden $n$. El número de elementos en S$_n$, es decir, de posibles permutaciones de $n$ números, es $n!$. 
		
		En los ejemplos musicales de este texto, los conjuntos estarán numerados desde 0 hasta $n-1$, siendo $n$ el número de elementos a permutar, en vez de desde 1 hasta $n$. Seguirán siendo grupos simétricos de orden $n$, pero con una numeración distinta.
		
		La notación utilizada para representar una permutación $\sigma$ perteneciente a S$_n$ con la numeración desde 0 y con $\sigma(m)$ siendo el elemento asociado a $m$ mediante $\sigma$, es:
		\[\sigma=\left(\begin{matrix}0&1&2&&n-3&n-2&n-1\\\sigma\left(0\right)&\sigma\left(1\right)&\sigma\left(2\right)&\cdots&\sigma\left(n-3\right)&\sigma\left(n-2\right)&\sigma\left(n-1\right)\\\end{matrix}\right)\]
		
	\subsection{Aritmética modular}
		Fijado un $n\in\mathbb{N}$, se dice que $a$ y $b$ son \emph{congruentes} (o equivalentes) módulo $n$ si tienen el mismo resto al dividirlos entre $n$; es decir, que todos los números con el mismo resto se agrupan y se toman como equivalentes. Se expresa como $a\equiv b$ (mod. $n$).
	
		De esta forma se pueden operar entre sí los números del 0 al $n-1$, ya que se conservan las operaciones de los números enteros, y si un resultado es $\geq n$ se puede seguir dividiendo entre $n$ para que cumpla $0\leq r<n$.
		
		Se conserva la suma (y la resta), ya que si $a=nq_a+r_a$ y $b=nq_b+r_b$, entonces $a+b=(nq_a+r_a)+(nq_b+r_b)=n(q_a+q_b)+(r_a+r_b)$, así que el resto de $a+b$ es igual al de $r_a+r_b$.
		
		La {aritmética modular} también se llama aritmética del reloj, porque funciona de la misma manera que las horas en un reloj. Como el 3 tiene el mismo resto entre 12 que el 15, las 15h son las 3h: $3\equiv15$ (mod. $12$). O, por ejemplo, 2 horas después de las 11 dan las 13, es decir, la 1: $2+11=13\equiv1$ (mod. $12$). 
		
		También se conserva la multiplicación: si $a=nq_a+r_a$ y $b=nq_b+r_b$, entonces $ab=(nq_a+r_a)(nq_b+r_b)=n^2q_aq_b+nq_ar_b+nq_br_a+r_ar_b=n(nq_aq_b+q_ar_b+q_br_a)+r_ar_b$, así que el resto de $ab$ es igual al de $r_ar_b$.
		
		En música, la aritmética modular se puede encontrar en las escalas: todas las notas Do se toman como equivalentes, por ejemplo, y al sumarle 12 semitonos (una octava) se vuelve a obtener un Do. Si se asocian los números del 0 al 11 a las notas cromáticas del Do al Si, entonces $0+12=12\equiv0$ (mod. $12$). Entonces se dice que un número $k$ pertenece al conjunto \{0, 1, 2, 3, 4, 5, 6, 7, 8, 9, 10, 11\}, con las propiedades indicadas, de esta manera: $k\in\mathbb{Z}_{12}$.%
	
	\renewcommand\refname{Referencias}
	\begin{thebibliography}{00}
			
%			\subsubsection*{\autoref{ch:filosofia}}
%			
			\bibitem{basomba}
			{ Basomba Garc\'ia, Daniel.} 
			\textit{El \'ultimo Bach y el dodecafonismo como ideal musical: una lectura est\'etica y sociol\'ogica},
			Universidad Carlos III de Madrid.
			Tesis Doctoral en Ciencia Pol\'itica y Sociolog\'ia
			(2013)
			
%			NADA
%			\bibitem{bhalerao}
%			{ Bhalerao, Rasika.} 
%			\textit{The Twelve-Tone Method of Composition},
%			Math 336.
%			Prof. Jim Morrow
%			(2015)
			
%			\bibitem{morris}
%			{ Morris, Robert.}
%			\textit{Mathematics and the Twelve-Tone System: Past, Present, and Future},
%			Perspectives of New Music 45.2 
%			(2007)
			
%			\bibitem{cook}
%			{ Cook, Nicholas.} Chapter 9: ``Analyzing Serial Music'',
%			\textit{A Guide to Musical Analysis},
%			New York: G. Braziller
%			(1987)

%			\bibitem{roberts}
%			{ Roberts, Gareth E.}
%			\textit{Composing with Numbers: Arnold Schoenberg and His Twelve-Tone Method},
%			Math/Music: Aesthetic Links
%			(2012).
						
	\end{thebibliography}

\end{document}
