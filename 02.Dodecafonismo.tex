\chapter[EL SISTEMA DODECAFÓNICO DE SCHOENBERG]{EL SISTEMA DODECAFÓ- NICO DE SCHOENBERG}
	\section{LOS POSTULADOS DEL DODECAFONISMO}
		El dodecafonismo es un sistema compositivo que predetermina la melodía y la armonía a partir de una ordenación de las doce notas de la escala cromática, que se llama \textit{serie}. Ésta y algunas de sus transformaciones son los ladrillos con los que se construye la obra: se deben colocar una detrás de otra. Son, por tanto, el único material que se puede utilizar para determinar las alturas de las notas.
		
		El resto de elementos de la pieza, como el número de instrumentos, el ritmo, el carácter, la textura o las dinámicas, se dejan a discreción del compositor. No serializar todos los conjuntos será la principal crítica al dodecafonismo por parte de los compositores serialistas que sucedieron a Schoenberg: los compositores de serialismo integral de mediados del siglo XX, como Pierre Boulez (apartado \ref{boulez}). Para los serialistas integrales, aquello restaba cohesión al modelo compositivo; para los dodecafonistas, aportaba libertad.
		
		Precisamente la predeterminación dodecafónica, aunque parece limitante, permite realizaciones musicales y estilos de composición muy diferentes: Schoenberg daba un tratamiento tradicional a sus obras, ya que aún admiraba las formas clásicas; Berg iba más allá al utilizar series que recordaban a las tríadas tonales; y, en cambio, Webern evitaba radicalmente cualquier asociación con la tradición.
		
		Schoenberg definió su sistema musical a partir de cuatro postulados que, en realidad, se basan en principios matemáticos:
		
		\emph{1. La serie \emph{[sobre la que se construye la obra dodecafónica]} consta de las doce notas de la escala cromática dispuestas en un orden lineal específico.}
		
		\emph{2. Ninguna nota aparece más de una vez en la serie.}
		
		Los dos primeros postulados expresan que una obra dodecafónica fundamenta su estructura sobre una permutación de la escala de doce semitonos. Dicha permutación $\sigma$ es una biyección del conjunto numerado de las doce notas \{Do = 0, Do\# = 1, Re = 2, Re\# = 3, Mi = 4, Fa = 5, F\# = 6, Sol = 7, Sol\# = 8, La = 9, La\# = 10, Si = 11\} consigo mismo, y se representa de esta forma:
		\begin{footnotesize}
			$$
			\sigma=\left(\begin{matrix}0&1&2&3&4&5&6&7&8&9&10&11\\\sigma(0)&\sigma(1)&\sigma(2)&\sigma(3)&\sigma(4)&\sigma(5)&\sigma(6)&\sigma(7)&\sigma(8)&\sigma(9)&\sigma(10)&\sigma(11)\\\end{matrix}\right)
			$$
		\end{footnotesize}	
		La permutación $\sigma(m)$, con $m\in \mathbb{Z} / (12)$\footnote{$\mathbb{Z} / (12)=\{0,\ 1,\ 2,\ 3,\ 4,\ 5,\ 6,\ 7,\ 8,\ 9,\ 10,\ 11\}$, el grupo cíclico de orden 12.}, pertenece al grupo simétrico de orden 12: $\sigma\in$ S$_{12}$. Por ejemplo, en la Suite para piano Op. 25 Schoenberg utiliza como serie original en todos los movimientos de la obra la siguiente permutación $\sigma$:
		$$\sigma=\left(\begin{matrix}0&1&2&3&4&5&6&7&8&9&10&11\\4&5&7&1&6&3&8&2&11&0&9&10\\\end{matrix}\right)$$	
		\includegraphics[width=138mm]{1.png}
		
		\emph{3. La serie será expuesta en cualquiera de sus aspectos lineales: original, inversión, retrogradación de la original y retrogradación de la inversión.}
		 
		\emph{4. La serie puede usarse en sus cuatro aspectos desde cualquier nota de la escala.}
		
		Los dos últimos postulados amplían los recursos compositivos al admitir la transformación de la serie original mediante \emph{inversión}, \emph{retrogradación}, \emph{inversión retrógrada} y \emph{transposición}\footnote{No confundir con un 2-ciclo. Una transposición musical se corresponde con una traslación matemática.}. El compositor puede utilizar cualquiera de las transformaciones de una serie al componer su obra dodecafónica. El conjunto de series que puede utilizar, que viene dado por la serie original y todas sus posibles transformaciones, se conoce como \emph{espectro serial}.
		
	\section{LAS TRANSFORMACIONES DE UNA SERIE}
		\label{transPsi}
		Transformar una serie es matemáticamente equivalente a aplicar una función sobre la serie que asocie su permutación a la permutación transformada. Por tanto, cualquier función transformativa $\Psi$ se aplica sobre el conjunto de las permutaciones: S$_{12}$, el grupo simétrico de orden 12 ($\Psi:\text{S}_{12}\rightarrow\text{S}_{12}$).
		
	\subsection{TRANSPOSICIONES}
		La \emph{transposición}, mencionada en el cuarto postulado, consiste en subir o bajar la serie original un número determinado de semitonos. Por tanto, no se modifican los intervalos entre las notas, sino solamente la altura a la que está la serie. Ya que consideraremos todas las octavas equivalentes, debemos trabajar módulo 12. 
		
		La serie transportada k semitonos, T$^\text{k}\left(\sigma\right)$, se construye sumando k a $\sigma$ (mod. 12):
		$$
		\forall m\in \mathbb{Z} / (12):	\text{T}^\text{k}\left(\sigma\left(m\right)\right)=\sigma\left(m\right)+\text{k} \qquad \text{con k constante;}
		$$		
		$$
		\text{T}^\text{k}=\left(\begin{matrix}0&1&2&&9&10&11\\\sigma\left(0\right)+\text{k}&\sigma\left(1\right)+\text{k}&\sigma\left(2\right)+\text{k}&\cdots&\sigma\left(9\right)+\text{k}&\sigma\left(10\right)+\text{k}&\sigma\left(11\right)+\text{k}\\\end{matrix}\right)
		$$
		
		A su vez, T$^\text{k}$ se forma al componer k transposiciones de 1 semitono, T$^1$: $\text{T}^\text{k}=\text{T}^1\circ\text{T}^1\circ\ldots\circ\text{T}^1$, k veces. Debido a que k es en realidad el exponente en la potencia de T, se coloca este número como superíndice.
		
		Históricamente, la notación $\Psi_\text{k}$, $\Psi^\text{k}$ o $\Psi(\text{k})$ se ha usado en sustitución de la composición de la transposición T$^\text{k}$ y otra función $\Psi$, en el respectivo orden: $\Psi^\text{k}=\Psi \circ \text{T}^\text{k} = \Psi(\text{T}^\text{k})$. Sin embargo, esta notación es especialmente ambigua y confusa. Por ello, es preferible ceñirse a la notación estrictamente matemática; es decir, a la composición de funciones, aun omitiendo el símbolo $\circ$.
		
		Una posible serie transportada sobre la permutación $\sigma$ de la Suite para piano Op. 25, con k $= 6$, es la siguiente serie T$^6$:
		$$\text{T}^6=\left(\begin{matrix}0&1&2&3&4&5&6&7&8&9&10&11\\10&11&1&7&0&9&2&8&5&6&3&4\\\end{matrix}\right)$$	
		\includegraphics[width=138mm]{2.png}
		
	\subsection{RETROGRADACIÓN}
		La \emph{retrogradación} consiste en leer la serie original desde la nota final hacia atrás, es decir, aplicar a la serie una simetría especular. De este modo, la primera nota irá al último puesto, la segunda al penúltimo, y así sucesivamente.
		
		La serie retrógrada se construye de esta forma:
		$$
		\forall m\in \mathbb{Z} / (12):	\text{R}\left(\sigma\left(m\right)\right)=\sigma\left(-1-m\right)$$ \begin{footnotesize}$$			\text{R}=\left(\begin{matrix}0&1&2&3&4&5&6&7&8&9&10&11\\	\sigma(11)&\sigma(10)&\sigma(9)&\sigma(8)&\sigma(7)&\sigma(6)&\sigma(5)&\sigma(4)&\sigma(3)&\sigma(2)&\sigma(1)&\sigma(0)\\\end{matrix}\right)
			$$\end{footnotesize}
			
		La serie retrógrada sobre la permutación $\sigma$ de la Suite Op. 25 es la siguiente serie R:	
		$$\text{R}=\left(\begin{matrix}0&1&2&3&4&5&6&7&8&9&10&11\\10&9&0&11&2&8&3&6&1&7&5&4\\\end{matrix}\right)$$		
		\includegraphics[width=138mm]{3.png}
		
	\subsection{INVERSIÓN}
		La \emph{inversión} consiste en cambiar la dirección --de ascendente a descendente, y viceversa-- de los intervalos entre cada nota de la serie. Si el primer intervalo en la serie original $\sigma$ es de $+k$, el primer intervalo en la serie invertida I será de $-k$ (mod. 12), por lo que debemos cambiar el signo de $\sigma$ para construir I. Además, queremos que la primera nota de ambas series, I(0) y $\sigma$(0), coincidan, así que debemos transportar la serie $-\sigma$ un número $\lambda$ de semitonos para que esta condición se cumpla:
		$$\text{I}(0)=-\sigma\left(0\right)+\lambda=\sigma\left(0\right)\implies \lambda=2\sigma(0)$$
		Por tanto, la serie invertida se construye de esta forma:
		$$
		\forall m\in \mathbb{Z} / (12):	\text{I}\left(\sigma\left(m\right)\right)=-\sigma\left(m\right)+2\sigma\left(0\right)
		$$
		\begin{footnotesize}	$$
		\text{I}=\left(\begin{matrix}0&1&2&&10&11\\\sigma(0)&-\sigma(1)+2\sigma(0)&-\sigma(2)+2\sigma(0)&\ldots&-\sigma(10)+2\sigma(0)&-\sigma(11)+2\sigma(0)\\\end{matrix}\right)
		$$	\end{footnotesize}
		
		La serie invertida sobre la permutación $\sigma$ de la Suite Op. 25 es la siguiente serie I:
		$$
		\text{I}=\left(\begin{matrix}0&1&2&3&4&5&6&7&8&9&10&11\\4&3&1&7&2&5&0&6&9&8&11&10\\\end{matrix}\right)
		$$		
		\includegraphics[width=138mm]{4.png}
				
		En total, obtendremos 48 series -- aunque no obligatoriamente distintas entre sí -- pertenecientes a un solo espectro serial. Hay 12 series originales sobre cada una de las doce notas, 12 series retrógradas, 12 invertidas y 12 series sobre las que se aplica tanto la retrogradación como la inversión. A continuación se muestra la nomenclatura histórica junto a la matemática:
		
		\begin{center}
		\begin{multicols}{2}
			\underline{Nomenclatura histórica}
			
			T$_0$, T$_1$, T$_2$\ldots
			
			R$_0$, R$_1$, R$_2$\ldots
			
			I$_0$, I$_1$, I$_2$\ldots
			
			IR$_0$, IR$_1$, IR$_2$\ldots
			
			\underline{Nomenclatura matemática}
			
			T$^0$, T$^1$, T$^2$\ldots
			
			R, RT$^1$, RT$^2$\ldots
			
			I, IT$^1$, IT$^2$\ldots
			
			IR, IRT$_1$, IRT$_2$\ldots
		\end{multicols}
		\end{center}

\begin{comment}

\subsection{GRUPO DE KLEIN DE LAS TRANSFORMACIONES}
\begin{wrapfigure}{L}{0.3\textwidth}
	\captionsetup{justification=centering, font=footnotesize}
	\vspace{-0.5cm}
	\centering{
		\includegraphics[width=0.22\textwidth]{Felix_Klein.jpeg}			
		\caption*{Felix Klein\\(1849--1925)}	}
\end{wrapfigure}
La retrogradación, la inversión y la composición de ambas son funciones involutivas; es decir, aplicando dos veces una transformación se vuelve a la serie original. Si tomamos las series transportadas como equivalentes, el conjunto de funciones restantes forma un grupo especial llamado grupo de Klein, donde la función identidad Id es el elemento identidad y RI equivale a IR. En general, un grupo de Klein es el formado por cuatro elementos donde cada elemento es inverso de sí mismo. \cite{bhalerao} \label{grupo4}

$$\text{Id}(\sigma)\circ \text{Id}(\sigma) = \text{Id}(\sigma)=\sigma$$
$$\text{R}(\sigma)\circ \text{R}(\sigma) = \text{R}(-1-\sigma) = -1-11+\sigma = \sigma$$
$$\text{I}(\sigma)\circ \text{I}(\sigma) = -\text{I}(\sigma)+2\text{I}(0) = -(-\sigma +2\sigma(0)) + 2\text{I}(0)=\sigma +2(\text{I}(0)-\sigma(0)) = \sigma + \text{k} \equiv \sigma$$		
$$\text{RI}(\sigma)\circ \text{RI}(\sigma) = -\text{RI}(-1-\sigma)+2\text{RI}(11) \equiv (-1-11+\sigma)+2\sigma(11)) = \sigma - 2\sigma(11) \equiv \sigma $$

El grupo de Klein, llamado así en honor al matemático alemán Felix Klein, es el grupo $\mathbb{Z}/(2)\times\mathbb{Z}/(2)$, producto directo de dos copias del grupo cíclico de orden 2.

\end{comment}
