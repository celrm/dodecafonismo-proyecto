\renewcommand\refname{Referencias}
	\begin{thebibliography}{00}
			
%			\subsubsection*{\autoref{ch:filosofia}}
%			
			\bibitem{basomba}
			{ Basomba Garc\'ia, Daniel.} 
			\textit{El \'ultimo Bach y el dodecafonismo como ideal musical: una lectura est\'etica y sociol\'ogica},
			Universidad Carlos III de Madrid.
			Tesis Doctoral en Ciencia Pol\'itica y Sociolog\'ia
			(2013)
			
%			NADA
%			\bibitem{bhalerao}
%			{ Bhalerao, Rasika.} 
%			\textit{The Twelve-Tone Method of Composition},
%			Math 336.
%			Prof. Jim Morrow
%			(2015)
			
%			\bibitem{morris}
%			{ Morris, Robert.}
%			\textit{Mathematics and the Twelve-Tone System: Past, Present, and Future},
%			Perspectives of New Music 45.2 
%			(2007)
			
%			\bibitem{cook}
%			{ Cook, Nicholas.} Chapter 9: ``Analyzing Serial Music'',
%			\textit{A Guide to Musical Analysis},
%			New York: G. Braziller
%			(1987)

%			\bibitem{roberts}
%			{ Roberts, Gareth E.}
%			\textit{Composing with Numbers: Arnold Schoenberg and His Twelve-Tone Method},
%			Math/Music: Aesthetic Links
%			(2012).
						
	\end{thebibliography}