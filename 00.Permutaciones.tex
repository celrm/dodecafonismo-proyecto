\addtocounter{chapter}{-1}%
\chapter{INTRODUCCIÓN MATEMÁTICA}
	\section{Conjuntos y grupos}
		Un \emph{conjunto} es una colección de objetos bien definidos y distintos entre sí que se llaman \emph{elementos}. 
	
		Para definir un conjunto se puede o bien listar los objetos uno a uno, o bien describirlos por medio de un predicado: una o varias propiedades que caracterizan a todos los elementos de dicho conjunto.

		Por ejemplo, el conjunto K$_\text{i}$, formado por las doce notas de la escala cromática de una misma octava i, está bien definido porque podemos hacer una lista con ellas: por ejemplo, $\text{K}_\text{4} = $
		
		\full{$\{\text{Do}_\text{4}, \text{Do\#}_\text{4}, \text{Re}_\text{4}, \text{Re\#}_\text{4}, \text{Mi}_\text{4}, \text{Fa}_\text{4}, \text{Fa\#}_\text{4}, \text{Sol}_\text{4}, \text{Sol\#}_\text{4}, \text{La}_\text{4}, \text{La\#}_\text{4}, \text{Si}_\text{4}\}$}
		
		Por un lado, aun llamando a las notas de distinta manera, el conjunto, conceptualmente, es el mismo. Además, el hecho de listar algún elemento más de una vez no afecta a su definición. Como $\text{Do\#}_\text{4} = \text{Re}\flat_\text{4}$,\footnote{En este texto se trabajará siempre con temperamento igual por convenio.}  $\text{K}_\text{4}$ también puede ser listado así:
		
		\full{$\{\text{Do}_\text{4}, \text{Do\#}_\text{4}, \text{Re}\flat_\text{4}, \text{Re}_\text{4}, \text{Re\#}_\text{4}, \text{Mi}_\text{4}, \text{Fa}_\text{4}, \text{Fa\#}_\text{4}, \text{Sol}_\text{4}, \text{Sol\#}_\text{4}, \text{La}_\text{4}, \text{La\#}_\text{4}, \text{Si}_\text{4}\}$}
	
		En cambio, el conjunto D, formado por las duraciones rítmicas elementales -- sin ligaduras ni puntillos --, es infinito, por lo que no se puede listar de forma completa. Sin embargo, se puede expresar por medio de un predicado:
		
		\full{$\text{D} =\{2^n:n\in\mathbb{Z},\ n\le 2\} = \{4,\ 2,\ 1,\ \frac{1}{2},\ \frac{1}{4},\ \frac{1}{8},\ \ldots\} = \{\fullnote,\ \halfnote,\ \quarternote,\ \eighthnote,\ \ldots\}$}
		
		La notación $n\in\mathbb{Z}$ significa que $n$ pertenece a los números enteros. En este caso se han representado las duraciones mediante su ratio con la duración de la negra $\quarternote$.
	
		Los elementos de un conjunto pueden combinarse mediante \emph{operaciones} -- como la suma o la multiplicación en el caso de los números -- para dar otros objetos matemáticos. 
		
		Se dice que un conjunto G no vacío y una operación binaria ($\ast$) forman la estructura de un \emph{grupo} (G, $\ast$) cuando cumplen:
	
		\begin{enumerate}
			\item{Su operación es interna: Si $a,b\in$ G, entonces $a\ast b\in$ G.}		
			\item{Su operación es asociativa: Si $a,b,c\in$ G, $(a\ast b)\ast c=a\ast(b\ast c)$. }
			\item{Existe un elemento $e$ en G, llamado elemento neutro o identidad, tal que para todo $x\in$ G se cumple que $e\ast x = x\ast e = x$. Se puede probar que el neutro es único para cada grupo. A veces se incluye dentro de la definición del grupo: (G, $\ast$, $e$).}
			\item{Cada $x\in$ G tiene asociado otro elemento $x^{-1}\in$ G, llamado elemento inverso, tal que $x \ast x^{-1} = x^{-1}  \ast x = e$. Se puede probar que el inverso de cada elemento es único.}		
		\end{enumerate}
	
	($\mathbb{Z},+,0$) y ($\mathbb{Q},+,0$) son grupos, pero ($\mathbb{N},+,0$) no porque no existe el \textit{inverso} de 2 con la suma: $-2\notin\mathbb{N}$. ($\mathbb{R},*,1$) y ($\mathbb{Q},*,1$) son grupos, pero ($\mathbb{Z},*,1$) no porque no existe el \textit{inverso} de 2 con la multiplicación: $\frac12\notin\mathbb{Z}$.
	
	\section{Funciones y permutaciones}
		Una \emph{función} es una regla que asocia a cada elemento de un primer conjunto, llamado \emph{dominio}, un único elemento de un segundo conjunto. Si la función se llama $f$, el dominio A y el segundo conjunto B, se denota $f:\text{A}\to \text{B}$. El elemento asociado a un $x$ mediante $f$ se denota $f(x)$.
		
		Todos los $x\in$ A tienen que estar asociados a un $f(x)\in$ B, pero no todos los elementos de B tienen un elemento de A asociado. Los elementos de B que sí lo cumplen, es decir, los que se pueden escribir como $f(x)$ para algún $x$, forman el conjunto \emph{imagen} de la función: $im(f)=\{\ y\in \text{B}:\ \exists\ x \in \text{A},\ f(x)=y\ \}$
		
		Cuando varias funciones se aplican una detrás de la otra decimos que realizamos la operación de \textit{composición de funciones}. Se representa con el símbolo $\circ$. La imagen de la primera función será el dominio de la segunda, y así sucesivamente. Por ejemplo, aplicar una función $f(x)$ y después aplicar una función $g(x)$ se denota $g(f(x))=(g\circ f)(x)$.
		
		Una \emph{permutación} $\sigma$(X) es una función sobre un conjunto X que asocia sus elementos a los elementos del mismo conjunto X de manera unívoca. Es decir, asocia cada elemento a uno, y solo uno, de los elementos de su mismo conjunto ($\sigma:\text{X}\to \text{X}$).

		El conjunto de todas las posibles permutaciones sobre un determinado conjunto X, junto con la operación de composición de funciones ($\circ$), forma un grupo denotado por S$_\text{x}$. Para probarlo, se debe comprobar que cumple todas las propiedades de los grupos.

		\begin{enumerate}
			\item{Permutar dos veces es también una permutación.}
			\item{La composición de funciones es asociativa.}
			\item{La permutación que asigna un elemento a sí mismo es la función identidad.}
			\item{Como las permutaciones son biyectivas, cada una tiene una inversa que es también una permutación.}		
		\end{enumerate}

		Cuando X es el conjunto de números naturales desde 1 hasta $n$, el grupo S$_\text{x}$ se representa como S$_n$ y se le denomina el grupo simétrico de orden $n$. El número de elementos en S$_n$, es decir, de posibles permutaciones de $n$ números, es $n!$. 
		
		En los ejemplos musicales de este texto, los conjuntos estarán numerados desde 0 hasta $n-1$, siendo $n$ el número de elementos a permutar, en vez de desde 1 hasta $n$. Seguirán siendo grupos simétricos de orden $n$, pero con una numeración distinta.
		
		La notación utilizada para representar una permutación $\sigma$ perteneciente a S$_n$ con la numeración desde 0 y con $\sigma(m)$ siendo el elemento asociado a $m$ mediante $\sigma$, es:
		\[\sigma=\left(\begin{matrix}0&1&2&&n-3&n-2&n-1\\\sigma\left(0\right)&\sigma\left(1\right)&\sigma\left(2\right)&\cdots&\sigma\left(n-3\right)&\sigma\left(n-2\right)&\sigma\left(n-1\right)\\\end{matrix}\right)\]
		
	\section{Aritmética modular básica}
		Fijado un $n\in\mathbb{N}$, se dice que $a$ y $b$ son \textit{congruentes} (o equivalentes) módulo $n$ si tienen el mismo resto al dividirlos entre $n$; es decir, que todos los números con el mismo resto se agrupan y se toman como equivalentes. Se expresa como $a\equiv b$ (mod. $n$).
	
		De esta forma se pueden operar entre sí los números del 0 al $n-1$, ya que se conservan las operaciones de los números enteros, y si un resultado es $\geq n$ se puede seguir dividiendo entre $n$ para que cumpla $0\leq r<n$.
		
		Se conserva la suma (y la resta), ya que si $a=nq_a+r_a$ y $b=nq_b+r_b$, entonces $a+b=(nq_a+r_a)+(nq_b+r_b)=n(q_a+q_b)+(r_a+r_b)$, así que el resto de $a+b$ es igual al de $r_a+r_b$.
		
		La \textit{aritmética modular} también se llama aritmética del reloj, porque funciona de la misma manera que las horas en un reloj. Como el 3 tiene el mismo resto entre 12 que el 15, las 15h son las 3h: $3\equiv15$ (mod. $12$). O, por ejemplo, 2 horas después de las 11 dan las 13, es decir, la 1: $2+11=13\equiv1$ (mod. $12$). 
		
		También se conserva la multiplicación: si $a=nq_a+r_a$ y $b=nq_b+r_b$, entonces $ab=(nq_a+r_a)(nq_b+r_b)=n^2q_aq_b+nq_ar_b+nq_br_a+r_ar_b=n(nq_aq_b+q_ar_b+q_br_a)+r_ar_b$, así que el resto de $ab$ es igual al de $r_ar_b$.
		
		En música, la aritmética modular se puede encontrar en las escalas: todas las notas Do se toman como equivalentes, por ejemplo, y al sumarle 12 semitonos (una octava) se vuelve a obtener un Do. Si se asocian los números del 0 al 11 a las notas cromáticas del Do al Si, entonces $0+12=12\equiv0$ (mod. $12$).