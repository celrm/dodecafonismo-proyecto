%\addtocounter{section}{-1}%
\section{INTRODUCCI\'ON MATEM\'ATICA}\label{ch:permutaciones}
	\subsection{Conjuntos y grupos}
		Un \emph{conjunto} es una colecci\'on de objetos bien definidos y distintos entre s\'i que se llaman \emph{elementos}. 
	
		Para definir un conjunto se puede o bien listar los objetos uno a uno, o bien describirlos por medio de un predicado: una o varias propiedades que caracterizan a todos los elementos de dicho conjunto.

		Por ejemplo, el conjunto K$_{\mbox{i}}$, formado por las doce notas de la escala crom\'atica de una misma octava i, est\'a bien definido porque podemos hacer una lista con ellas: por ejemplo, $\mbox{K}_{\mbox{4}} = $
		
		{$\{\mbox{Do}_{\mbox{4}}, \mbox{Do\#}_{\mbox{4}}, \mbox{Re}_{\mbox{4}}, \mbox{Re\#}_{\mbox{4}}, \mbox{Mi}_{\mbox{4}}, \mbox{Fa}_{\mbox{4}}, \mbox{Fa\#}_{\mbox{4}}, \mbox{Sol}_{\mbox{4}}, \mbox{Sol\#}_{\mbox{4}}, \mbox{La}_{\mbox{4}}, \mbox{La\#}_{\mbox{4}}, \mbox{Si}_{\mbox{4}}\}$}
		
		Por un lado, aun llamando a las notas de distinta manera, el conjunto, conceptualmente, es el mismo. Adem\'as, el hecho de listar alg\'un elemento m\'as de una vez no afecta a su definici\'on. Como $\mbox{Do\#}_{\mbox{4}} = \mbox{Re}\flat_{\mbox{4}}$,\footnote{En este texto se trabajar\'a siempre con temperamento igual por convenio.}  $\mbox{K}_{\mbox{4}}$ tambi\'en puede ser listado as\'i:
		
		{$\{\mbox{Do}_{\mbox{4}}, \mbox{Do\#}_{\mbox{4}}, \mbox{Re}\flat_{\mbox{4}}, \mbox{Re}_{\mbox{4}}, \mbox{Re\#}_{\mbox{4}}, \mbox{Mi}_{\mbox{4}}, \mbox{Fa}_{\mbox{4}}, \mbox{Fa\#}_{\mbox{4}}, \mbox{Sol}_{\mbox{4}}, \mbox{Sol\#}_{\mbox{4}}, \mbox{La}_{\mbox{4}}, \mbox{La\#}_{\mbox{4}}, \mbox{Si}_{\mbox{4}}\}$}
	
		En cambio, el conjunto D, formado por las duraciones r\'itmicas elementales -- sin ligaduras ni puntillos --, es infinito, por lo que no se puede listar de forma completa. Sin embargo, se puede expresar por medio de un predicado:
		
		{$\mbox{D} =\{2^n:n\in\mathbb{Z},\ n\le 2\} = \{4,\ 2,\ 1,\ \dfrac{1}{2},\ \dfrac{1}{4},\ \dfrac{1}{8},\ \ldots\} = \{\fullnote,\ \halfnote,\ \quarternote,\ \eighthnote,\ \ldots\}$}
		
		La notaci\'on $n\in\mathbb{Z}$ significa que $n$ pertenece a los n\'umeros enteros. En este caso se han representado las duraciones mediante su ratio con la duraci\'on de la negra $\quarternote$.
	
		Los elementos de un conjunto pueden combinarse mediante \emph{operaciones} -- como la suma o la multiplicaci\'on en el caso de los n\'umeros -- para dar otros objetos matem\'aticos. 
		
		Se dice que un conjunto G no vac\'io y una operaci\'on binaria ($\ast$) forman la estructura de un \emph{grupo} (G, $\ast$) cuando cumplen:
	
		\begin{enumerate}
			\item{Su operaci\'on es interna: Si $a,b\in$ G, entonces $a\ast b\in$ G.}		
			\item{Su operaci\'on es asociativa: Si $a,b,c\in$ G, $(a\ast b)\ast c=a\ast(b\ast c)$. }
			\item{Existe un elemento $e$ en G, llamado elemento neutro o identidad, tal que para todo $x\in$ G se cumple que $e\ast x = x\ast e = x$. Se puede probar que el neutro es \'unico para cada grupo. A veces se incluye dentro de la definici\'on del grupo: (G, $\ast$, $e$).}
			\item{Cada $x\in$ G tiene asociado otro elemento $x^{-1}\in$ G, llamado elemento inverso, tal que $x \ast x^{-1} = x^{-1}  \ast x = e$. Se puede probar que el inverso de cada elemento es \'unico.}		
		\end{enumerate}
	
	($\mathbb{Z},+,0$) y ($\mathbb{Q},+,0$) son grupos, pero ($\mathbb{N},+,0$) no porque no existe el \textit{inverso} de 2 con la suma: $-2\notin\mathbb{N}$. ($\mathbb{R},*,1$) y ($\mathbb{Q},*,1$) son grupos, pero ($\mathbb{Z},*,1$) no porque no existe el \textit{inverso} de 2 con la multiplicaci\'on: $\dfrac12\notin\mathbb{Z}$.
	
	\subsection{Funciones y permutaciones}
		Una \emph{funci\'on} es una regla que asocia a cada elemento de un primer conjunto, llamado \emph{dominio}, un \'unico elemento de un segundo conjunto. Si la funci\'on se llama $f$, el dominio A y el segundo conjunto B, se denota $f:\mbox{A}\to \mbox{B}$. El elemento asociado a un $x$ mediante $f$ se denota $f(x)$.
		
		Todos los $x\in$ A tienen que estar asociados a un $f(x)\in$ B, pero no todos los elementos de B tienen un elemento de A asociado. Los elementos de B que s\'i lo cumplen, es decir, los que se pueden escribir como $f(x)$ para alg\'un $x$, forman el conjunto \emph{imagen} de la funci\'on: $im(f)=\{\ y\in \mbox{B}:\ \exists\ x \in \mbox{A},\ f(x)=y\ \}$
		
		Cuando varias funciones se aplican una detr\'as de la otra decimos que realizamos la operaci\'on de \textit{composici\'on de funciones}. Se representa con el s\'imbolo $\circ$. La imagen de la primera funci\'on ser\'a el dominio de la segunda, y as\'i sucesivamente. Por ejemplo, aplicar una funci\'on $f(x)$ y despu\'es aplicar una funci\'on $g(x)$ se denota $g(f(x))=(g\circ f)(x)$.
		
		Una \emph{permutaci\'on} $\sigma$(X) es una funci\'on sobre un conjunto X que asocia sus elementos a los elementos del mismo conjunto X de manera un\'ivoca. Es decir, asocia cada elemento a uno, y solo uno, de los elementos de su mismo conjunto ($\sigma:\mbox{X}\to \mbox{X}$).

		El conjunto de todas las posibles permutaciones sobre un determinado conjunto X, junto con la operaci\'on de composici\'on de funciones ($\circ$), forma un grupo denotado por S$_{\mbox{x}}$. Para probarlo, se debe comprobar que cumple todas las propiedades de los grupos.

		\begin{enumerate}
			\item{Permutar dos veces es tambi\'en una permutaci\'on.}
			\item{La composici\'on de funciones es asociativa.}
			\item{La permutaci\'on que asigna un elemento a s\'i mismo es la funci\'on identidad.}
			\item{Como las permutaciones son biyectivas, cada una tiene una inversa que es tambi\'en una permutaci\'on.}		
		\end{enumerate}

		Cuando X es el conjunto de n\'umeros naturales desde 1 hasta $n$, el grupo S$_{\mbox{x}}$ se representa como S$_n$ y se le denomina el grupo sim\'etrico de orden $n$. El n\'umero de elementos en S$_n$, es decir, de posibles permutaciones de $n$ n\'umeros, es $n!$. 
		
		En los ejemplos musicales de este texto, los conjuntos estar\'an numerados desde 0 hasta $n-1$, siendo $n$ el n\'umero de elementos a permutar, en vez de desde 1 hasta $n$. Seguir\'an siendo grupos sim\'etricos de orden $n$, pero con una numeraci\'on distinta.
		
		La notaci\'on utilizada para representar una permutaci\'on $\sigma$ perteneciente a S$_n$ con la numeraci\'on desde 0 y con $\sigma(m)$ siendo el elemento asociado a $m$ mediante $\sigma$, es:
		\[\sigma=\left(\begin{matrix}0&1&2&&n-3&n-2&n-1\\\sigma\left(0\right)&\sigma\left(1\right)&\sigma\left(2\right)&\cdots&\sigma\left(n-3\right)&\sigma\left(n-2\right)&\sigma\left(n-1\right)\\\end{matrix}\right)\]
		
	\subsection{Aritm\'etica modular}
		Fijado un $n\in\mathbb{N}$, se dice que $a$ y $b$ son \textit{congruentes} (o equivalentes) m\'odulo $n$ si tienen el mismo resto al dividirlos entre $n$; es decir, que todos los n\'umeros con el mismo resto se agrupan y se toman como equivalentes. Se expresa como $a\equiv b$ (mod. $n$).
	
		De esta forma se pueden operar entre s\'i los n\'umeros del 0 al $n-1$, ya que se conservan las operaciones de los n\'umeros enteros, y si un resultado es $\geq n$ se puede seguir dividiendo entre $n$ para que cumpla $0\leq r<n$.
		
		Se conserva la suma (y la resta), ya que si $a=nq_a+r_a$ y $b=nq_b+r_b$, entonces $a+b=(nq_a+r_a)+(nq_b+r_b)=n(q_a+q_b)+(r_a+r_b)$, as\'i que el resto de $a+b$ es igual al de $r_a+r_b$.
		
		La \textit{aritm\'etica modular} tambi\'en se llama aritm\'etica del reloj, porque funciona de la misma manera que las horas en un reloj. Como el 3 tiene el mismo resto entre 12 que el 15, las 15h son las 3h: $3\equiv15$ (mod. $12$). O, por ejemplo, 2 horas despu\'es de las 11 dan las 13, es decir, la 1: $2+11=13\equiv1$ (mod. $12$). 
		
		Tambi\'en se conserva la multiplicaci\'on: si $a=nq_a+r_a$ y $b=nq_b+r_b$, entonces $ab=(nq_a+r_a)(nq_b+r_b)=n^2q_aq_b+nq_ar_b+nq_br_a+r_ar_b=n(nq_aq_b+q_ar_b+q_br_a)+r_ar_b$, as\'i que el resto de $ab$ es igual al de $r_ar_b$.
		
		En m\'usica, la aritm\'etica modular se puede encontrar en las escalas: todas las notas Do se toman como equivalentes, por ejemplo, y al sumarle 12 semitonos (una octava) se vuelve a obtener un Do. Si se asocian los n\'umeros del 0 al 11 a las notas crom\'aticas del Do al Si, entonces $0+12=12\equiv0$ (mod. $12$).