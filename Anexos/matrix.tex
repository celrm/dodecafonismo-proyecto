\usepackage{ifthen}
\usepackage{tcolorbox}
\usepackage{xparse}
\usepackage{tikz}
\usetikzlibrary{matrix}

\ExplSyntaxOn
\DeclareExpandableDocumentCommand{\eval}{m}{\int_eval:n {#1}}
\ExplSyntaxOff

\newcounter{first}
\newcounter{doce}
\setcounter{first}{-1}

\newcommand{\md}[1]{
	\ifnum#1>\thedoce
		\ \eval{#1-\thedoce}
	\else\ifnum#1<0
		\ \eval{#1+\thedoce}
	\else #1
	\fi\fi
}
\newcommand{\row}[2]{%
	\foreach \i in {#1} {%
		\begingroup\edef\x{\endgroup
			\noexpand\gappto\noexpand\mymatrixcontent{ 
				{#2}\&
		}}\x
	}%
	\gappto\mymatrixcontent{\\}%
}
\newcommand{\mat}[1]{
	\let\mymatrixcontent\empty
	\setcounter{doce}{0}
	\setcounter{first}{-1}
	
	\foreach \n in {#1}{
		\ifnum\thefirst=-1
			\setcounter{first}{\n}
		\fi
		\stepcounter{doce}
	}
	
	\foreach \j in {#1}{
		\row{#1}{%
			\md{\eval{\i-\j+\thefirst}}
		}
	}

	\resizebox{\linewidth}{!}{
		\begin{tikzpicture}
		\matrix [ampersand replacement=\&, 
			matrix of math nodes, 
			row sep=0.2em, column sep=1em]
		{\mymatrixcontent};
		\end{tikzpicture}
	}
}