	\chapter{Código para el cálculo de matrices dodecafónicas.}
	\label{app:matrices}
		
	\lstset{
		tabsize=3,
		language=C++,
		frame=lines,
		numbers=left,
		identifierstyle=\color{magenta},
%			
%		numberstyle=\tiny,
		numberstyle=\ttfamily\tiny\color[gray]{0.3},		
%		basicstyle=\footnotesize,
		basicstyle=\small\sffamily,
%		keywordstyle=\color[rgb]{0,0,1},
		keywordstyle=\bfseries\rmfamily,
%		commentstyle=\color[rgb]{0.09, 0.45, 0.27},
		commentstyle=\it,
%		stringstyle=\color{red}
		stringstyle=\mdseries\rmfamily,
%		
		xleftmargin=2pt,
		stepnumber=1,
		numbersep=5pt,
		belowcaptionskip=\bigskipamount,
		captionpos=b,
		escapeinside={*'}{'*},
		emphstyle={\bf},
		showspaces=false,
		columns=flexible,
		showstringspaces=false,
	}
		
		\newpage
	\begin{lstlisting}
	#include <iostream>
	using namespace std;
	
	const int N = 12;
	
	int main() {
	
		int s[N + 2][N + 2];
		
		for (int i = 1; i < N + 1; ++i) {
			cin >> s[1][i];
			s[i][0] = (N - s[1][i] + s[1][1]) % N;
			s[i][N + 1] = s[i][0];
			s[0][i] = (N - s[i][0]) % N;
			s[N + 1][i] = s[0][i];
		}
		
		for (int i = 2; i < N + 1; ++i)
			for (int j = 1; j < N + 1; ++j)
				s[i][j] = (s[1][j] + s[i][0]) % N;
		
		cout << "\n$$\\begin{array}{l|";
		for (int i = 0; i < N; ++i)	cout << 'c';
		cout << "|r}\n&";
		
		for (int i = 1; i < N + 1; ++i)
			cout << "\\text{I}_{" << s[0][i] << "}&";
		cout << "\\\\\n\\hline\n";
		
		for (int i = 1; i < N + 1; ++i) {
			cout << "\\text{T}_{" << s[i][0] << "}&";
		
			for (int j = 1; j < N + 1; ++j)
				cout << s[i][j] << "&";
			
			cout << "\\text{R}_{" << s[i][N + 1] << "}\\\\\n";
		}
		cout << "\\hline\n&";
		
		for (int i = 1; i < N + 1; ++i)
			cout << "\\text{IR}_{" << s[N + 1][i] << "}&";
		cout << "\\\\\n\\hline\n&";
		
		const int DIF = 2 * (s[1][N] - s[1][1]);
		for (int i = 1; i < N + 1; ++i)
			cout << "\\text{RI}_{" << (N + s[0][i] + DIF) % N << "}&";
		
		cout << "\n\\end{array}$$\n";
		
		return 0;
	}
	
	\end{lstlisting}