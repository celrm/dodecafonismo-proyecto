\documentclass[a4, 11pt]{article}

\usepackage{microtype}
%\usepackage[T1]{fontenc} % dejar solo para html
\usepackage[latin1]{inputenc}

\textwidth = 6.5 in 
\textheight = 9 in 
\oddsidemargin = 0.0 in
\evensidemargin = 0.0 in 
\topmargin = 0.0 in
\headheight = 0.0 in 
\headsep = 0.0 in
\parindent = 0.3in
\parskip=10pt
 
\usepackage{graphicx}
\graphicspath{{Fotos/}}

\usepackage{amsmath}
\usepackage{amsfonts}
\usepackage{harmony}
\usepackage{enumerate}
\usepackage{enumitem}
\usepackage{bm}
\usepackage{upgreek}
\usepackage{cancel}

%\input{ddphonism.sty}

\usepackage{multicol}
\PassOptionsToPackage{hyphens}{url}
\usepackage{hyperref}

% Complex \xxx for making notes of things to do.  Use \xxx{...} for general
% notes, and \xxx[who]{...} if you want to blame someone in particular.
% Puts text in brackets and in bold font, and normally adds a marginpar
% with the text ``xxx'' so that it is easy to find.  On the other hand, if
% the comment is in a minipage, figure, or caption, the xxx goes in the text,
% because marginpars are not possible in these situations.
{\makeatletter
 \gdef\xxxmark{%
   \expandafter\ifx\csname @mpargs\endcsname\relax % in minipage?
     \expandafter\ifx\csname @captype\endcsname\relax % in figure/caption?
       \marginpar{xxx}% not in a caption or minipage, can use marginpar
     \else
       xxx % notice trailing space
     \fi
   \else
     xxx % notice trailing space
   \fi}
 \gdef\xxx{\@ifnextchar[\xxx@lab\xxx@nolab}
 \long\gdef\xxx@lab[#1]#2{{\bf [\xxxmark #2 ---{\sc #1}]}}
 \long\gdef\xxx@nolab#1{{\bf [\xxxmark #1]}}
 % This turns them off:
% \long\gdef\xxx@lab[#1]#2{}\long\gdef\xxx@nolab#1{}%
}

\renewcommand\refname{Bibliograf�a}
\renewcommand\figurename{Figura}

\title{Re-escalando m�sica}
\author{Celia Rubio Madrigal}
\date{}

\begin{document}
\maketitle

Tras la serie de art�culos {\it Serialismo y matem�ticas} que ha ido de  septiembre a noviembre de 2019, Celia Rubio, su autora, me present� un cuarto texto, {\it Re-escalando m�sica}. Este texto constituye la columna de diciembre de 2019. Es de nuevo una inmensa fortuna contar con la colaboraci�n de Celia Rubio. Dejamos al lector con el placer de su escritura y concepto.

%Este art�culo es el tercero y �ltimo de la colecci�n {\it Serialismo y matem�ticas}. Las m�sicas serialistas son aquellas que permiten construir castillos con un solo grano de arena: una serie particular, una permutaci�n de notas, din�micas o timbres. La serie se coloca en la obra secuencialmente, siempre igual o con alguna modificaci�n que la adorne. Y es que para esta m�sica,  la serie es el ladrillo y las matem�ticas son la pintura con la que decorarlos, ya que las transformaciones que se le puede aplicar a una serie forman preciosas estructuras matem�ticas enmarcadas en la Teor�a de Grupos.

%En el primer art�culo~\cite{celia1} nos centramos en el dodecafonismo, en sus or�genes y en comentar una de sus obras. En el segundo art�culo~\cite{celia2} ampliamos las definiciones dodecaf�nicas para encontrar el grupo di�drico, y descubrimos la historia de los disc�pulos de Schoenberg y del serialismo integral.

%Esta tercera entrega est� destinada al lector m�s ducho en las matem�ticas; se notar� en el lenguaje y en la exposici�n de las ideas. En ella proporcionaremos herramientas matem�ticas relacionadas con \textbf{acciones}, \textbf{�rbitas} y \textbf{estabilizadores} de Teor�a de Grupos (\hyperref[ch:acciones]{secci�n 2}), para despu�s contar de dos maneras distintas el \textbf{n�mero de espectros seriales}, que son el n�mero de �rbitas del grupo de transformaciones sobre las series, que un compositor puede utilizar en sus obras (\hyperref[ch:espectros]{secci�n 3}); en concreto, con las transformaciones $\{I,\ T,\ R\}$ (\hyperref[s:itr]{3.1}) y $\{S,\ T,\ V,\ C\}$ (\hyperref[s:stvc]{3.2}). Y de esta manera habremos hecho un recorrido a fondo por el serialismo y habremos explorado sus posibilidades musicales y matem�ticas.
	
	
    \section{La visi�n art�stica de Schoenberg}
    En julio de 1921, tras haber ideado los fundamentos del dodecafonismo, Schoenberg anunci� a su disc�pulo Josef Rufer \cite{quote1}:
    \begin{quote}
    	\textit{He realizado un descubrimiento que asegurar� la supremac�a de la m�sica alemana durante los pr�ximos cien a�os.}
    \end{quote}

    Durante la mayor parte de su vida, Schoenberg crey� que el p�blico general acabar�a aceptando la m�sica dodecaf�nica del mismo modo que se hab�an aceptado los sistemas tonales desde hac�a siglos. Para �l, la naturalidad del sistema dodecaf�nico resid�a en que era un paso m�s en el proceso musical hist�rico: desde el contrapunto y el desarrollo mot�vico, practicado por los grandes maestros de la tradici�n alemana, hasta la disoluci�n de la tonalidad, anticipada por la m�sica postwagneriana e impresionista. Era parte de un continuo, del desarrollo de la historia de la m�sica. En palabras de Schoenberg \cite{scho}:
    \begin{quote}
    	\textit{Yo creo que la composici�n con doce sonidos y la que muchos llaman err�neamente ``m�sica atonal'', no es el final de un viejo per�odo, sino el comienzo de otro nuevo. Una vez m�s, como hace dos siglos, hay algo a lo que se llama anticuado; y una vez m�s, no se trata de ninguna obra en particular, [...] % ni de varias obras de determinado compositor; de nuevo, no es la mayor o menor maestr�a de tal compositor, 
    	sino que otra vez sucede que es un estilo el condenado al ostracismo. %Vuelve a darse a s� misma la denominaci�n de M�sica Nueva [...]
    }
    \end{quote}

	Tras la muerte de Schoenberg en 1951, y durante algunas d�cadas m�s, su sistema compositivo fue venerado por los compositores j�venes m�s brillantes (v�ase \cite{celia2}), pero pronto se desvaneci� de las salas de conciertos. El serialismo siempre se consider� una m�sica acad�mica, dif�cil de entender, apenas musical sino te�rica. La complejidad de percibir esta m�sica meramente por su estructura formal impidi�, y todav�a impide, que se disfrutara m�s all� de su estudio. Schoenberg intent� eximirse culpando al oyente, quien �l crey� que no se esforzaba lo suficiente \cite{scho}:
	
	\begin{quote}
		\textit{La composici�n con doce sonidos no tiene otra finalidad que la comprensi�n.
			A la vista de ciertos acontecimientos en la historia musical reciente, 
			�sto puede causar asombro, ya que las obras escritas en este estilo no han sido entendidas %a pesar del nuevo medio de organizaci�n. %Por lo que, si nos olvid�ramos de que nuestros contempor�neos no son los �ltimos jueces, sino que la historia es generalmente la que predomina, habr�amos de considerar condenado este m�todo. Pero, si bien parece aumentar las dificultades para el oyente, �sto se compensa con las penalidades del compositor. Porque no resulta f�cil el componer de esta forma, sino diez veces m�s dif�cil; 
			[...] Solo el compositor perfectamente preparado ser� quien componga para el oyente musical igualmente bien dispuesto.}
	\end{quote}
	
	Al contrario de lo que Schoenberg cre�a, incluso el \textit{oyente experto}, el que describe T. W. Adorno en su ``Introducci�n a la sociolog�a de la m�sica'' \cite{adorno}, tiene grandes dificultades para distinguir auditivamente todos los elementos que caracterizan el serialismo. Somos capaces de retener, a lo sumo, motivos de seis o siete notas, pero no de doce \cite{sevenpm2}; mucho menos de reconocer si una serie es transformaci�n de otra. �En qu� medida afectan las reglas dodecaf�nicas al discurso sonoro de una pieza?
	
	El dodecafonismo puede atribuirse el haber prescindido de algunas de las preconcepciones musicales m�s arraigadas, como la melod�a, la consonancia o la tonalidad. Pero precisamente por eso es impopular, porque toma la disonancia y la pone al frente de toda la composici�n. Para Schoenberg, la aprobaci�n del p�blico no era el objetivo de su arte, y, de hecho, el desagrado colectivo era un signo del alto nivel art�stico y espiritual al que se encontraba \cite{scho}:
	
%	\begin{quote}
%		\emph{La belleza es una necesidad de los mediocres.}%\footnote{A. Schoenberg, \emph{Harmonielehre}, 1922.}
%	\end{quote}
	\begin{quote}
		\emph{El valor de mercado es irrelevante para el valor intr�nseco. Un juicio no cualificado puede como m�ximo decidir el valor de mercado \textemdash un valor que puede ser inversamente proporcional al valor intr�nseco.}%\footnote{A. Schoenberg, \emph{An Artistic Impression} (1909) en \emph{Style and Idea}, 1985.}
	\end{quote}
	\begin{quote}
		\emph{Ning�n artista, ning�n poeta, ning�n fil�sofo y ning�n m�sico, cuyo pensamiento se desenvuelve en la m�s alta esfera, habr� de descender a la vulgaridad para mostrarse complacientes con un eslogan tal como ``Arte para todos''. Porque si es arte no ser� para todos, y si es para todos no ser� arte.}%\footnote{A. Schoenberg, \emph{New Music, Outmoded Music, Style and Idea}, 1946.}
	\end{quote}
	
	Sin embargo, el rechazo a no ser rechazado ha dejado de tener cabida en nuestro contexto art�stico. El academicismo ya no es excluyente a la divulgaci�n o a la b�squeda de belleza sensorial. De las t�cnicas serialistas se puede tomar aquello que es interesante intelectualmente e incorporarlo a otras t�cnicas.
	
	Este es el experimento que he querido proponer: despojar al serialismo de uno de los elementos que provoca m�s rechazo: la disonancia extrema. Ya que esta proviene del cromatismo, el prop�sito del experimento es utilizar escalas que tengan menos intervalos de semitono para crear con ellas un \textit{pseudo-serialismo} de menos notas. 
	
	Se han modificado las notas de varias obras dodecaf�nicas ya existentes, mientras que el ritmo, la duraci�n, el timbre y las din�micas, que siguen siendo producto de los compositores originales, se han dejado intactas. El prop�sito final es intentar conservar la estructura matem�tica subyacente renovando, en cambio, la percepci�n colectiva de estas m�sicas.
	
	Para describir el proceso de modificaci�n de las obras debemos definir lo que se entiende por escala y cu�les son las funciones �ptimas entre escalas.%
	\section{Escalas y funciones del experimento}\label{ch:escalas}	
	\subsection{Escalas interv�licas, escalas y funciones}
	
		Una \textbf{escala interv�lica} es una secuencia ordenada de n�meros naturales -- una secuencia de intervalos entre notas -- tales que la suma de todos ellos da 12. As� solo consideramos v�lidas las escalas equivalentes octava a octava%: todos los intervalos de la escala deben sumar el n�mero de semitonos de una octava
		. Esto debe ocurrir para poder considerar transformaciones de la escala crom�tica en escalas menores, aunque es generalizable a cualquier longitud. Diremos entonces que la escala crom�tica es la \textbf{s�per-escala} de las \textbf{sub-escalas} con las que trabajaremos. Por ejemplo, la escala diat�nica j�nica (o escala mayor) tiene como secuencia (2, 2, 1, 2, 2, 2, 1).
		
		Dada una escala interv�lica de longitud $\ell$ y una nota fija inicial, la secuencia de intervalos se convierte en una secuencia de notas de longitud $\ell$+1. Se construye comenzando por la nota inicial y sumando cada intervalo para conseguir la nota siguiente.
		
		Con la escala mayor y la nota Re se consigue (Re, Mi, Fa$\#$, Sol, La, Si, Do$\#$, Re), ya que es equivalente a (2, 2+\textbf{2}=4, 4+\textbf{2}=6, 6+\textbf{1}=7, 7+\textbf{2}=9, 9+\textbf{2}=11, 11+\textbf{2}=13, 13+\textbf{1}=14). Por construcci�n, la �ltima nota debe ser equivalente a la primera, ya que en el �ltimo paso habremos sumado a la nota inicial todos los t�rminos de la secuencia interv�lica, y por definici�n suman 12.
		
		De esta forma, se puede definir una \textbf{escala-\textit{k}} como el conjunto de notas generadas por una escala interv�lica desde la nota $k$. Por ejemplo, el conjunto anterior ser�a la escala-2 mayor; es decir, la escala de Re mayor. Una escala generada por una secuencia de intervalos con longitud $\ell$ tiene $\ell$ notas, ya que como la �ltima es repetida no hay por qu� considerarla. Su longitud $\ell\leq 12$, ya que una escala-\textit{k} definida de esta forma siempre es un subconjunto de la escala crom�tica: $E_k\subseteq\mathbb{Z}/(12)$. Al generalizarlo a cualquier s�per-escala, habr�a que considerar las notas distintas seg�n su escala o formular otras definiciones m�s adecuadas.
		
		Una \textbf{funci�n a una escala-\textit{k}} es una funci�n $f$ que transforma cada nota de la escala crom�tica a un valor de la escala $E_k$. Entonces $f : \mathbb{Z}/(12) \rightarrow E_k^*$ reduce las notas de una melod�a a solamente la escala escogida, donde $E_k^*$ est� formado por las notas de $E_k$ pero quiz�s en octavas distintas. Las funciones a escalas se representan de la siguiente manera, con la primera fila representando el dominio de $f$ (la escala crom�tica); la segunda su imagen (la escala con repeticiones y en distintas octavas, $E_k^*$); y la tercera su secuencia interv�lica, que es de inter�s, ya que coincide con la escala interv�lica de partida salvo en los valores nulos.
		
		\[\left.\begin{matrix}
		0&1&2&\ldots&10&11\\
		f(0)&f(1)&f(2)&\ldots&f(10)&f(11)\\
		f(1)-f(0)&f(2)-f(1)&f(3)-f(2)&\ldots&f(11)-f(10)&12+f(0)-f(11)
		\end{matrix}\right.\]
		
%		En realidad, la $k$ de la escala-$k$ no es especialmente relevante, porque una escala-$(k+1)$ es la transportada de una $k$. Se puede escoger sin p�rdida de generalidad $k=0$ a partir de ahora, y as� todas comenzar�n en Do.
		
		El proceso verdaderamente interesante est� en averiguar, dada una escala $E$, cu�l es la mejor funci�n que transforma melod�as crom�ticas en melod�as en $E$. Estas son las \textbf{funciones \textit{E}-inducidas}.
		
		�Cu�les ser�n las caracter�sticas de esas funciones �ptimas? Deben ser sobreyectivas: si no, la m�sica resultante tendr�a una escala m�s reducida de la deseada. Pero adem�s deben conservar la estructura serial y deben conservar el parecido con la melod�a original. 
		
	\subsection{Funciones bien distribuidas}
		
		La mayor prioridad es conservar la estructura serial de las piezas; por tanto, todas las notas deben aparecer con la menor frecuencia posible, y se debe evitar jerarqu�as entre las notas en la medida de lo posible. Si $|E|<12$, $f$ no puede ser inyectiva, por lo que va a haber elementos repetidos en la imagen. Queremos la $f$ que mejor distribuya esas repeticiones, que distribuya las notas de $E$ a lo largo de la escala crom�tica.
		
		Lo �ptimo ser�a que todas tuvieran la misma frecuencia. Eso solo pasar� cuando $|E|$ divida a 12. Por ejemplo, si $E = \{a_1,a_2,a_3,a_4,a_5,a_6\}$ (entonces $|E|=6$), existen funciones tales que cada nota de la imagen se repite exactamente 2 veces. La siguiente funci�n $E$-inducida $f$ cumplir�a la condici�n de buena distribuci�n:
				
		\[\left.\begin{array}{*{13}c}
		\text{Crom�tica}&0&1&2&3&4&5&6&7&8&9&10&11\\
		\text{Escala }E&a_1&a_1&a_2&a_2&a_3&a_3&a_4&a_4&a_5&a_5&a_6&a_6\\
		\text{Intervalos}&&&&&&\ldots\\
		\end{array}\right.\]
		
		En cambio, si $|E|$ no divide a 12 no hay funciones $E$-inducidas totalmente distribuidas. No existe una sola frecuencia que puedan compartir todas las notas de $E$. Sin embargo, s� se pueden encontrar dos frecuencias consecutivas, $c$ y $c+1$, tales que todos los elementos de $E$ tengan o frecuencia $c$ o frecuencia $c+1$. Esto es lo m�s parecido a que todas tengan la misma frecuencia, y se va a probar a continuaci�n que siempre es posible. 
		
		La situaci�n es equivalente a que $E$ se pueda dividir en dos subconjuntos disjuntos $Q$ y $R$, con $|Q|=q$ y $|R|=r$ (entonces $q+r=|E|$), tales que la frecuencia de las notas en Q es $c$ y la frecuencia de las notas en $R$ es $c+1$. En resumen, para probar que $Q$ y $R$ existen, debemos encontrar un $c$, un $q$ y un $r$ naturales para los que $cq + (c+1)r=12$.
		
		$cq + (c+1)r =
		cq + cr + r =
		c(q+r) + r =
		c|E| + r = 12$, lo cual se cumple por el algoritmo de la divisi�n, que asegura que al dividir 12 entre $|E|$ existen su cociente $c$ y su resto $r\geq0$. $\square$
		
		La siguiente tabla describe, para cada posible $|E|$ en cada fila, la frecuencia �ptima de sus elementos. Las columnas representan las frecuencias de los elementos, y los n�meros de dentro son cada $q$ y $r$ (cuando es 0 no se escribe: no hay notas con esa frecuencia).
	
		\[\begin{array}{l|rrrrrrrrrrrr}
		&1&2&3&4&5&6&7&8&9&10&11&12\\\hline
		1&&&&&&&&&&&&1\\\hline
		2&&&&&&2\\\hline
		3&&&&3\\\hline
		4&&&4\\\hline
		5&&3&2\\\hline
		6&&6\\\hline
		7&2&5\\\hline
		8&4&4\\\hline
		9&6&3\\\hline
		10&8&2\\\hline
		11&10&1\\\hline
		12&12&\\
		\end{array}\]
		
		Estas funciones forman parte del numeroso conjunto de elementos musicales de \textbf{m�xima regularidad}. Un ejemplo importante de ellos son los ritmos eucl�deos \textemdash para m�s informaci�n ver \cite{euclides}.
		
	\subsection{Funciones \textit{E}-inducidas}
		
		Hay que pedir m�s requisitos a $f$ para que no solo modifique las notas, sino que adem�s las im�genes se parezcan lo m�ximo posible a sus preim�genes, a las notas originales. En esencia, lo que se busca es una escala a \textbf{distancia m�nima} de la escala crom�tica en cuanto a unos criterios concretos.
		
		La manera matem�tica de formalizar esos criterios es definir una \textbf{m�trica} para estas funciones; es decir, una manera de medir la distancia entre ellas para poder compararlas. La distancia $d$ entre dos funciones $f$ y $g$ cualesquiera, $d(f,g)$, debe cumplir estas propiedades b�sicas:
		\begin{multicols}{2}
			\begin{enumerate}
				\item $d(f,g)>0$
				\item $d(f,g)=0 \Longleftrightarrow f=g$
				\item $d(f,g)=d(f,g)$
				\item $d(f,g)\leq d(f,h)+d(h,g)$
			\end{enumerate}
		\end{multicols}
		
		La m�trica que he escogido para comparar las funciones consiste en restar sus im�genes una a una, tomar el valor absoluto de esas diferencias y sumarlas: $d(f,g)=\sum\limits_{i=0}^{11} |f(i)-g(i)|$. Esto nos da una idea de c�mo de ``lejos'' se encuentran una de la otra, y cumple los axiomas de una m�trica.
		
		Nos interesa entonces encontrar la funci�n m�s cercana a la funci�n identidad, es decir, la que enviar�a la escala crom�tica a sus mismas notas. As� se priorizan las funciones con el mayor n�mero de puntos fijos \textemdash ya que el sumando en ese �ndice ser�a 0\textemdash~, o que, al menos, se parezcan en su escala interv�lica asociada.
		
		Puede ocurrir que con esta manera de medir quede m�s de una funci�n a distancia m�nima. Entre ellas, yo he escogido la m�s grave, y as�, dada cualquier escala $E$, su funci�n $E$-inducida queda un�vocamente determinada.
		
		En el enlace \url{https://gitlab.com/dodecafonismo/f-inducida} se encuentra el c�digo en Haskell de un programa que, dado una escala, produce su funci�n inducida �ptima con las propiedades descritas anteriormente.
		
		En el c�digo se puede escoger entre o bien encontrar la mejor funci�n que use solamente las notas de la subescala dada, o bien permitir transposiciones de �sta \textemdash que conservan, aun as�, la escala interv�lica asociada\textemdash~y que es a lo que llamo ``inducir la ra�z''. Tambi�n permite cambiar el dominio, o superescala, y que no sea la crom�tica, aunque en ese caso puede que la m�trica definida no devuelva resultados tan intuitivos.
		
	\section{Modificaci�n de partituras serialistas}\label{ch:modif}
\subsection{Escalas utilizadas}

Las escalas escogidas para este experimento son cuatro escalas de distintos tama�os y sonoridades; desde el sonido oriental hasta el occidental cl�sico, pasando por el jazz moderno y el impresionismo.
Son la escala pentat�nica, la escala de tonos enteros, la escala heptaf�nica de do mayor y la escala octot�nica. Estas son las funciones inducidas de dichas escalas seg�n el algoritmo:

\[\left.\begin{array}{*{13}c}
\text{Crom�tica:}&0&1&2&3&4&5&6&7&8&9&10&11\\
\text{Pentat�nica (5):}&0&0&2&2&4&4&7&7&7&9&9&12\\
\text{Intervalos:}&&2&&2&&3&&&2&&3&\\
\end{array}\right.\]		
\[\left.\begin{array}{*{13}c}
\text{Crom�tica:}&0&1&2&3&4&5&6&7&8&9&10&11\\
\text{Tonos enteros (6):}&0&0&2&2&4&4&6&6&8&8&10&10\\
\text{Intervalos:}&&2&&2&&2&&2&&2&&2\\
\end{array}\right.\]
\[\left.\begin{array}{*{13}c}
\text{Crom�tica:}&0&1&2&3&4&5&6&7&8&9&10&11\\
\text{Diat�nica en do (7):}&0&0&2&2&4&5&5&7&7&9&9&11\\
\text{Intervalos:}&&2&&2&1&&2&&2&&2&1\\
\end{array}\right.\]        
\[\left.\begin{array}{*{13}c}
\text{Crom�tica:}&0&1&2&3&4&5&6&7&8&9&10&11\\
\text{Octot�nica (8):}&0&0&2&3&3&5&6&6&8&9&9&11\\
\text{Intervalos:}&&2&1&&2&1&&2&1&&2&1\\
\end{array}\right.\]

	\subsection{Obras modificadas}
	
	Ahora se describir�n las obras que pasar�n por la modificaci�n. Para abarcar distintos estilos compositivos y hacer este estudio m�s amplio, he escogido obras de los tres principales compositores dodecaf�nicos: Schoenberg, Berg y Webern.
	
	Sin embargo, no se han escogido obras de compositores posteriores ni serialistas integrales. Uno de los motivos es porque interesa en este estudio la relaci�n entre los sonidos: no se modifican m�s que las alturas de las notas, y por tanto no se tiene en cuenta el resto de elementos musicales. Que est�n compuestos serialmente no afecta a las conclusiones de este experimento.
	
	Por otro lado, los compositores posteriores a Schoenberg todav�a no han pasado al dominio p�blico. Eso impide, por desgracia, que se pueda trabajar libremente con su m�sica.
	
	Por �ltimo, el hecho de que cada nota tenga su propia din�mica, su propia articulaci�n o su propio timbre hace de las obras serialistas integrales dif�ciles de manipular. Adem�s, como los audios est�n hechos mediante ordenador y no con int�rpretes reales, la calidad y la intenci�n musical de estas partituras tan complicadas nunca podr�an plasmarse a la perfecci�n.
	
	La primera obra que pasar� por el algoritmo de modificaci�n serial es la \textit{Suite para piano}, Op. 25 de Schoenberg. Un an�lisis de esta pieza y de su contexto hist�rico se puede encontrar en \cite{celia1}.
	
	Su \href{http://www.ccarh.org/publications/data/humdrum/tonerow/files/schoenberg/schoenberg04.pc.krn}{serie principal} es: \drow{4,5,7,1,6,3,8,2,11,0,9,10}
\begin{center}
		\ddiagram{4,5,7,1,6,3,8,2,11,0,9,10}
\end{center}
	
	La segunda obra es un arreglo para soprano y piano de una de las arias m�s destacadas de la segunda �pera de Alban Berg, \textit{Lulu}. El libreto de la obra est� basado en dos tragedias de Frank Wedekind: ``El esp�ritu de la tierra'' y ``La Caja de Pandora''.
	
	El aria, llamada \textit{Lied der Lulu}, es parte de una dram�tica disputa entre Lulu y su marido por las infidelidades de ella, que acaba con el homicidio accidental de �l.
	
	La \href{http://www.ccarh.org/publications/data/humdrum/tonerow/files/berg/berg10.pc.krn}{serie} de Lulu es:
		\drow{0,4,5,2,7,9,6,8,11,10,3,1}
\begin{center}
		\ddiagram{0,4,5,2,7,9,6,8,11,10,3,1}
\end{center}
	
	La tercera, de 1936, es la �nica obra publicada de Anton Webern para piano solo: \textit{Variationen f�r Klavier}, Op. 27, y se compone de tres movimientos: \textit{Sehr m�ssig}, \textit{Sehr schnell} y \textit{Ruhig fliessend}.
	
	Su \href{http://www.ccarh.org/publications/data/humdrum/tonerow/files/webern/webern17.pc.krn}{serie principal} es:
		\drow{3,11,10,2,1,0,6,4,7,5,9,8}
\begin{center}
		\ddiagram{3,11,10,2,1,0,6,4,7,5,9,8}
\end{center}
    
    \subsection{Programa \textit{online} de modificaci�n de partituras}
	He creado una p�gina web \textit{online} que transforma cada nota de una partitura a cualquier nota requerida, una a una. Este \textit{software} sirve para no tener que modificar a mano las partituras del experimento, pero tambi�n puede servir para otros prop�sitos. Por ejemplo, para cambiar una partitura de mayor a menor, o viceversa.
	
	\includegraphics[width=15cm]{whiteweb2.jpg}
	
	El programa solo admite partituras con formato \textit{Archivo Musescore sin Comprimir} (\textit{.mscx}) del software libre \href{https://musescore.org/}{\textit{Musescore}}. En caso de tener la partitura en otro formato, debe abrirse en \textit{Musescore} y guardarse en el formato correcto. Est� escrita en Elm y el c�digo puede encontrarse \href{https://gitlab.com/dodecafonismo/modificaciones}{aqu�}.
	
   	 La aplicaci�n web se encuentra en el siguiente enlace: \url{https://modificaciones.netlify.com/}. Sus instrucciones de uso se encuentran al final de la p�gina web.
    
    \section{Conclusiones}
    Todas las conclusiones que se pueden extraer de este experimento son enteramente subjetivas. El objetivo de realizarlo es poder seguir investigando con las propiedades matem�ticas de la m�sica, y analizar el impacto emocional que �stas pueden causar.
    
    No se puede afirmar que la transformaci�n mejore o empeore ninguna obra. En todo caso podemos interpretar qu� transformaciones tienen un determinado sentido musical o est�tico, dependiendo de la escala utilizada o del estilo con el que est�n compuestas.     
    Tampoco debemos olvidar que el cromatismo siempre aportar� a las obras una dimensi�n a�adida, un elemento extra que ha impulsado gran parte de la innovaci�n en la historia de la m�sica. Quitarlo por completo es, en realidad, retroceder en la evoluci�n del arte.
   	
   	En general, las transformaciones hexat�nica y octot�nica siguen conservando mucho del cromatismo que tiene la partitura original. Siguen sonando ajenas al o�do tonal del oyente medio. Vamos a comentar algunas de las impresiones que generan las otras dos transformaciones en cada una de las obras, aunque dejaremos al lector que forme su propia opini�n.
   	
    \subsection{Obra de Berg: Lied der Lulu}
	El estilo compositivo de Berg busca, en su mayor parte, acercarse a las formas tonales; maneja la falta de tonalidad serialista sin deshacerse de muchos elementos de la tradici�n musical. Sus melod�as son fluidas y su fraseo inicia a conversar. As�, la transformaci�n pentat�nica (5) queda, quiz�s, algo simplista y repetitiva, y es en cambio la heptat�nica (7) la que nos traslada a sonoridades m�s familiares.
	
    \url{https://soundcloud.com/celiarubio/sets/berg-lied-der-lulu}
    
    \subsection{Obra de Webern: Variationen op. 27}
    El estilo compositivo de Webern es rompedor y enigm�tico. Tanto fue as� que su m�sica sirvi� de inspiraci�n para el serialismo integral de los a�os 50. Sus melod�as suenan fragmentadas y est�n llenas de intervalos de m�s de una octava. Es, por tanto, muy dif�cil que cualquier transformaci�n que conserve similitudes mel�dicas con la partitura original pueda acercarse a m�sicas m�s convencionales. La esencia de esta obra est� precisamente en su peculiaridad.
    
    \url{https://soundcloud.com/celiarubio/sets/webern-op27-variations}
    
    
    \subsection{Obra de Schoenberg: Suite op. 25}
    El estilo compositivo de Schoenberg en la Suite es tradicional, aunque busca nuevas sonoridades. Su principal objetivo es conservar la estructura formal anterior, y por ello lo �nico que aleja a la obra es el uso del serialismo en la altura de las notas.
    
    La obra es, en general, m�s arm�nica que mel�dica, ya que pretende simular texturas instrumentales del periodo barroco. Adem�s, al centrarse tanto en la formalidad de la pieza aporta una riqueza separada del uso del serialismo. Por ello, la transformaci�n pentat�nica (5) no acaba siendo mon�tona sino muy sugestiva.
    
    Por otro lado, la elecci�n concreta de la funci�n transformativa, que hace predominar las notas do y sol \textemdash que aparecen en la nueva serie una vez m�s que el resto de notas\textemdash~provoca que, en muchos casos, la obra simule estar en do mayor. Como en la partitura original predomina el intervalo de tritono re $\flat$ \textendash~sol, la transformaci�n da peso al intervalo de quinta justa, que es la base de la armon�a tradicional.
    
    La transformaci�n heptat�nica sigue dejando alguna disonancia debido a la existencia de semitonos entre las notas mi \textendash~fa y si \textendash~ do, y al tritono en fa \textendash~si. Al ser una obra ampliamente textural, muchos de estos intervalos aparecen con frecuencia.
    
    \url{https://soundcloud.com/celiarubio/sets/schoenberg-op25-1-prelude}
    
    \url{https://soundcloud.com/celiarubio/sets/schoenberg-op25-2a-gavotte}
    
    \url{https://soundcloud.com/celiarubio/sets/schoenberg-op25-2b-musette}
    
    \url{https://soundcloud.com/celiarubio/sets/schoenberg-op25-3-intermezzo}
    
    \url{https://soundcloud.com/celiarubio/sets/schoenberg-op25-4a-menuet}
    
    \url{https://soundcloud.com/celiarubio/sets/schoenberg-op25-4b-trio}
    
    \url{https://soundcloud.com/celiarubio/sets/schoenberg-op25-5-gigue}

	
\bibliographystyle{plain}
\bibliography{Re-escalando-musica.bib}
\end{document}





