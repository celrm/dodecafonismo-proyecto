	\chapter{OBRAS Y ESCALAS DEL EXPERIMENTO}
    % 09.Obras y escalas a utilizar y su estudio: por qué suena mal la hexafónica.
    \section{OBRAS A MODIFICAR}
    	El objetivo de este experimento es modificar algunas obras que ya están compuestas mediante el método dodecafónico, y cambiar su serialismo de doce notas por otro pseudoserialismo de menos notas. Para abarcar distintos estilos compositivos y hacer este estudio más riguroso, se han escogido obras de los tres principales compositores dodecafónicos: Schoenberg, Berg y Webern.
        
        Sin embargo, no se han escogido obras de compositores posteriores ni serialistas integrales. Uno de los motivos es porque interesa en este estudio la relación entre los sonidos: no se modifican más que las alturas de las notas, y por tanto no importa el resto de elementos musicales. Que estén compuestos serialmente no afecta a las conclusiones de este estudio.
        
        Por otro lado, los compositores posteriores a Schoenberg todavía no han pasado al dominio público. Eso impide, por desgracia, que se pueda trabajar libremente con su música.
        
        Por último, el hecho de que cada nota tenga su propia dinámica, su propia articulación o su propio timbre hace de las obras serialistas integrales que sean difíciles de manipular. Además, como los audios están hechos mediante ordenador y no con intérpretes reales, la calidad y la intención musical de estas partituras tan complicadas nunca pueden plasmarse a la perfección.
        
    	\subsection{SCHOENBERG: \textit{Suite para piano}, OP. 25}
        	La primera obra que pasará por el algoritmo de modificación serial es la Suite para piano de Schoenberg, ya comentada en el capítulo \ref{suitechapter}. En dicha sección se estudió su serie principal, la estructura general de la obra y su razón de ser, así como su tercer movimiento en profundidad.
            
            http://www.ccarh.org/publications/data/humdrum/tonerow/files/schoenberg/schoenberg04.pc.krn
            
    	\subsection{BERG: \textit{Lied der Lulu}}
        	La gran calidad emocional de la música de Alban Berg se refleja en esta segunda obra: es una de las arias más destacadas de su segunda ópera, \textit{Lulu}. El arreglo a voz y piano fue realizado por Erwin Stein, un músico austriaco amigo y discípulo de Schoenberg.
        	
        	El libreto de la obra está basado en dos tragedias de Frank Wedekind: ``El espíritu de la tierra'' y ``La Caja de Pandora''. 
        	
        	%argumento de la ópera
        	
        	
        	
        	%series utilizadas (por personaje)

		BERG: LULU: PRIMARY ROW
		
		{0,4,5,2,7,9,6,8,B,A,3,1} (Jarman)
		
		http://www.ccarh.org/publications/data/humdrum/tonerow/files/berg/berg10.pc.krn
            
        % argumento de esta aria
        
        \subsection{BERG: \textit{Der Wein}}
        
        BERG: DER WEIN
        
        {2,4,5,7,9,A,1,6,8,0,B,3}
        
        http://www.ccarh.org/publications/data/humdrum/tonerow/files/berg/berg09.pc.krn
        
        Der Wein (The Wine) (1929), Concert Aria for Soprano and Orchestra
        
        Alban Berg's concert aria Der Wein (1929) is a setting of three poems from Charles Baudelaire's ``Le Vin'' as translated into German by Stephan George: "Die Seele des Weines" (The Wine's Soul), "Der Wein der Liebenden" (The Wine of Lovers), "Der Wein des Einsamen" (The Wine of the Lonely One). The poems express not only the happiness and confidence (real or imagined) of those who enjoy wine's restorative powers, but also wine's celebration of itself as a giver of strength to the weak, pride to the poor, and inspiration to poets.

		Appropriately, Berg's setting is lush, evoking earthly sensations through the use of jazz elements and tango rhythms in the manner so popular in Europe in the 1920s. (Compare, for example, the contemporaneous music of Kurt Weill and Paul Hindemith.) The principal twelve-tone row Berg fashioned for Der Wein lends itself to the construction of diatonic sonorities: the first six pitches comprise most of a D minor scale, while of the row's remaining six, five are part of a G flat major scale. Few of Berg's works place as apparent an emphasis on symmetry as does Der Wein. There are no breaks between the poems, and the two outer songs provide the exposition and recapitulation of a sonata-form movement. ``Die Seele des Weines'' consists of an orchestral introduction, first and second theme groups with a transition, and closing material. All of this music, albeit abbreviated, returns in "Der Wien des Einsamen" in the same order and is followed by a coda. Where a traditional development section might be expected, Berg instead interpolates the scherzo-like "Der Wein der Liebenden." This middle song itself falls into three sections, the third of which is a retrograde of the second. Extended palindromes of this sort reappear in Berg's opera Lulu (1935), notably in the film music of Act II.

		Der Wein was commissioned by Ruzena Herlinger, who gave the first performance of the work in Frankfurt on June 4, 1930.
		
		\subsection{WEBERN: \textit{Variations}, OP. 27}     
		
		WEBERN: OP. 27--VARIATIONS FOR PIANO
		
		{3,B,A,2,1,0,6,4,7,5,9,8}
		
		\subsection{WEBERN: \textit{3 Lieder}, OP. 18}
		
		WEBERN: OP. 18, NO. 1, "SCHATZERL KLEIN"
		
		{0,B,5,8,A,9,3,4,1,7,2,6}
		
		WEBERN: OP. 18, NO. 2, "ERLOSUNG"
		
		{6,9,5,8,4,7,3,B,2,A,1,0}
				
		WEBERN: OP. 18, NO. 3, "AVE, REGINA COELORUM"
		
		{4,3,7,6,5,B,A,2,1,0,9,8}	
		
	\section{ESCALAS Y FUNCIONES A UTILIZAR}
	
		Una \textit{escala interválica} es una secuencia ordenada de números naturales -- una secuencia de intervalos entre notas -- tales que la suma de todos ellos da 12. Así solo se consideran válidas las escalas equivalentes octava a octava: todos los intervalos de la escala deben sumar el número de semitonos de una octava. Por ejemplo, la escala diatónica jónica (o escala mayor) tiene como secuencia (2, 2, 1, 2, 2, 2, 1).
		
		%En general, se puede considerar que dos escalas son esencialmente iguales si sus secuencias comienzan desde puntos distintos de la misma secuencia. Por ejemplo, todas las escalas diatónicas son esencialmente iguales.
		
		Dada una escala interválica de longitud L y una nota fija inicial, la secuencia de intervalos se plasma en una secuencia de notas de longitud L+1. Se construye comenzando por la nota inicial y sumando cada intervalo para conseguir la nota siguiente. Con la escala mayor y la nota Re se consigue \{Re, Mi, Fa$\#$, Sol, La, Si, Do$\#$\}, ya que es equivalente a \{2, 2+\textbf{2}=4, 4+\textbf{2}=6, 6+\textbf{1}=7, 7+\textbf{2}=9, 9+\textbf{2}=11, 11+\textbf{2}=13, 13+\textbf{1}=14\}. Por construcción, la última nota debe ser equivalente a la primera, ya que en el último paso habremos sumado a la nota inicial todos los términos de la secuencia interválica, que por definición da 12.
		
		De esta forma, se puede definir una \textit{escala-k} como el conjunto de notas generadas por una escala interválica desde la nota $k$. Por ejemplo, el conjunto anterior sería la escala-2 mayor; es decir, la escala de Re mayor. Una escala generada por una secuencia de intervalos con longitud L tiene L notas.
		
		Este experimento pretende que se transformen las notas de la escala cromática en notas de otras escalas reducidas. Por ello se define una \textit{función a una escala-k} E$_k$ es una función $f$ que relaciona cada nota de la escala cromática con un valor de la escala E$_k$. Entonces $f : \mathbb{Z}/(12) \rightarrow \text{E}_k$ reduce las notas de una melodía cromática a solamente la escala escogida.
		
		El proceso verdaderamente interesante está en averiguar, dada una escala E$_k$, cuál es la mejor función a E$_k$. La mayor prioridad es conservar la estructura serial de las piezas; por tanto, hay que intentar que todas las notas aparezcan con la menor frecuencia posible y evitar la jerarquía entre las notas lo máximo posible. Si $|\text{E}|<12$, $f$ no puede ser inyectiva. Queremos la $f$ que mejor distribuya las notas de E$_k$ a lo largo de la escala cromática. 
		
		El siguiente gráfico describe, para cada posible tamaño de escala en cada fila, la frecuencia óptima de sus elementos. Para una escala E con tamaño $t$, la columna $t\ |\ a_1\ a_2\ a_3\ \ldots\ a_{12}$ cumple:
		
		$\sum_{r=1}^{12}a_r=t$, porque cada nota de E debe aparecer en una sola columna.
		
		$\sum_{r=1}^{12}a_r*r=12$, porque las columnas indican cuántas veces aparece en la imagen
		
		$$\begin{array}{l|rrrrrrrrrrrr}
		&1&2&3&4&5&6&7&8&9&10&11&12\\\hline
		1&&&&&&&&&&&&1\\\hline
		2&&&&&&2\\\hline
		3&&&&3\\\hline
		4&&&4\\\hline
		5&&3&2\\\hline
		6&&6\\\hline
		7&2&5\\\hline
		8&4&4\\\hline
		9&6&3\\\hline
		10&8&2\\\hline
		11&10&1\\\hline
		12&12&\\
		\end{array}$$
		
		Si $f$ no fuera sobreyectiva, la música resultante utilizaría una escala más reducida de la deseada. Hay que pedir más requisitos a $f$ para que no solo modifique las notas, sino que además se parezcan lo máximo posible a sus preimágenes, a las notas originales. Por ejemplo:
		
		\begin{enumerate}[i)]
			\item{$f$ debe ser creciente: si $n<m\in\mathbb{Z}/(12)$ entonces $f(n)\leq f(m)$. Si no lo fuera, aquellas dos notas decrecientes se deberían intercambiar.}
			\item{
					
			}
		\end{enumerate}
		
		Las funciones a escalas se representan de la siguiente manera:
		
		$$\left.\begin{matrix}
		\text{Escala cromática:}&0&1&2&3&4&5&6&7&8&9&10&11\\
		\text{Escala E}&f(0)&f(1)&f(2)&f(3)&f(4)&f(5)&f(6)&f(7)&f(8)&f(9)&f(10)&f(11)\\
		\text{Intervalos}&&2&&2&1&&2&&2&&2&1\\
		\end{matrix}\right.$$

        Randomly generated function
        \begin{lstlisting}
        #include <iostream>
        #include <cstdlib>
        #include <algorithm>
        
        using namespace std;
        using VI = int[12];
        
        int main() {
        	srand (time(NULL));
        	VI v;
        	for (size_t i = 0; i < 12; i++)
        		v[i] = rand()%12;
        	sort(v, v+12);
        	for (size_t i = 0; i < 12; i++)
        		cout << v[i] << " ";
        	cout << "\n";
        	return 0;
        }        
        \end{lstlisting}

        Chromatic dodecaphonic scales
        \begin{lstlisting}
        #include <iostream>
        #include <cstdlib>
        
        using namespace std;
        using VI = int[12];
        
        int main() {
        	srand (time(NULL));
        	VI v = {0,1,2,3,4,5,6,7,8,9,10,11};
        	int a,b;
        	for (size_t i = 0; i < 24; i++) {
        		a = rand() % 12;
        		b = v[i%12];
        		v[i%12] = v[a];
        		v[a] = b;
        	}
        	for (size_t i = 0; i < 12; i++)
        		cout << v[i] << " ";
        	cout << "\n";
        	return 0;
        }        
        \end{lstlisting}

        $$\left.\begin{matrix}
        \text{Escala cromática:}&0&1&2&3&4&5&6&7&8&9&10&11\\
        \text{Escala diatónica en Do:}&0&0&2&2&4&5&5&7&7&9&9&11\\
        \text{Intervalos}&&2&&2&1&&2&&2&&2&1\\
        \end{matrix}\right.$$

        $$\left.\begin{matrix}
        \text{Escala cromática:}&0&1&2&3&4&5&6&7&8&9&10&11\\
        \text{Escala de tonos enteros:}&0&0&2&2&4&4&6&6&8&8&10&10\\
        \text{Intervalos}&&2&&2&&2&&2&&2&&2\\
        \end{matrix}\right.$$

        $$\left.\begin{matrix}
        \text{Escala cromática:}&0&1&2&3&4&5&6&7&8&9&10&11\\
        \text{Escala pentatónica mayor:}&0&0&2&2&4&4&7&7&7&9&9&0\\
        \text{Intervalos}&&2&&2&&3&&&2&&3&\\
        \end{matrix}\right.$$
        
        $$\left.\begin{matrix}
        \text{Escala cromática:}&0&1&2&3&4&5&6&7&8&9&10&11\\
        \text{Escala octotónica:}&0&0&2&3&3&5&6&6&8&9&9&11\\
        \text{Intervalos}&&2&1&&2&1&&2&1&&2&1\\
        \end{matrix}\right.$$
