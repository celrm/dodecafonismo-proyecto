\renewcommand\refname{Referencias}
	\begin{thebibliography}{00}
%		\addcontentsline{toc}{part}{BIBLIOGRAF\'IA}
			
%			\bibitem{wright}
%			{ Wright, David.} 
%			\textit{Mathematics and Music},
%			American Mathematical Society 
%			(2009).			
			
%			\subsubsection*{\autoref{ch:historia}}
			\bibitem{mha}
			{ de Aixquivel, J.} 
			\textit{Memorias de Historia Antigua},
			Universidad de Oviedo.
			(1989)
			\\\url{https://books.google.es/books?id=m6zZrpltkzYC}
			
			\bibitem{jeppesen}
			{ Jeppesen, Knud} 
			\textit{Counterpoint: the polyphonic vocal style of the sixteenth century}.
			(1931)
			\\\url{https://archive.org/details/counterpointpoly00jepp}
			
			\bibitem{kinney}
			{ Kinney, James P.} 
			\textit{Twelve-tone Serialism: Exploring the Works of Anton Webern},
			University of San Diego.
			Undergraduate Honors Theses.
			(2015)
			\\\url{https://digital.sandiego.edu/honors_theses/1}
			
			\bibitem{diaz}
			{ D\'iaz de la Fuente, Alicia.} 
			\textit{Estructura y significado en la m\'usica serial y aleatoria},
			Universidad Nacional de Educaci\'on a Distancia.
			Tesis Doctoral en Filosof\'ia.
			(2005)
			\\\url{https://dialnet.unirioja.es/servlet/tesis?codigo=38135}		
			
%			\subsubsection*{\autoref{ch:dodecafonismo}}
			
%			\bibitem{delgado}
%			{ Delgado Garc\'ia, Fernando.} 
%			Clases y material de Historia de la M\'usica, 5$^{\circ}$ y 6$^{\circ}$ de Ense\~nanzas Profesionales del Conservatorio Profesional de M\'usica Arturo Soria, cursos 2014-15 y 2015-16.
			
			\bibitem{boulez}
			{ Boulez, Pierre.}
			\textit{Schoenberg is dead},
			The Score.
			(1952)
			\\\url{http://www.ubu.com/papers/Boulez-Schoenberg+Is+Dead.pdf}
			
			\bibitem{dominguez}
			{ Dom\'inguez Romero, Manuel.} 
			\textit{Las Matem\'aticas en el Serialismo Musical},
			Sigma n.24, 93-98.
			(2004)
			\\\url{http://www.hezkuntza.ejgv.euskadi.eus/r43-573/es/contenidos/informacion/dia6_sigma/es_sigma/adjuntos/sigma_24/6_Serialismo_musical.pdf}

%			\subsubsection*{\autoref{ch:suite}}
			
			\bibitem{ilomaki}
			{ Ilom\"aki, Tuukka.}
			\textit{On the Similarity of Twelve-Tone Rows},
			Sibelius Academy.
			(2008)
			\\\url{https://helda.helsinki.fi/handle/10138/235041}
			
			\bibitem{hyde}
			{ Hyde, Martha.} Chapter 4: ``Dodecaphonism: Schoenberg'',
			\textit{Models of Musical Analysis: Early Twentieth-century Music},
			Ed. Mark Everist and Jonathan Dunsby.
			Oxford: Blackwell.
			(1993)
			\\\url{https://books.google.es/books?id=JSdVHAAACAAJ}
					
			\bibitem{xiao}
			{ Xiao, June.} 
			\textit{Bach's Influences in the Piano Music of Four 20th Century Composers},
			Indiana University Jacobs School of Music.
			Doctoral Theses in Music.
			(2014)
			\\\url{https://scholarworks.iu.edu/dspace/handle/2022/19212}
			
			\bibitem{clercq}
			{ Clercq, Trevor de.} 
			\textit{A Window into Tonality via the Structure of Schoenberg's ``Musette'' from the Piano Suite, op. 25},
			Theory/Analysis of 20th-Century Music.
			(2006)
			\\\url{http://www.midside.com/pdf/eastman/fall06/th513/schoenberg_op25_analysis.pdf}
			
%			\subsubsection*{\autoref{ch:grupo}}			
%			
%			\bibitem{hunter}
%			{ Hunter, David J.; von Hippela, Paul T.}
%			\textit{How Rare Is Symmetry in Musical 12-Tone Rows?},
%			The American Mathematical Monthly, Vol. 110, No. 2
%			(2003).
%			
%			\subsubsection*{\autoref{ch:acciones}}
%			
%			\bibitem{armstrong}
%			{ Armstrong, M. A.} Chapter 6: ``Permutations'', Chapter 17: ``Actions, Orbits, and Stabilizers'', Chapter 18: ``Counting Orbits'',
%			\textit{Groups and Symmetry},
%			New York: Springer-Verlag
%			(1988)
%			
%			\subsubsection*{\autoref{ch:espectros}}
%			
%			\bibitem{polygons}
%			{ Golomb, S. W., Welch, L. R.}
%			\textit{On the enumeration of polygons},
%			The American Mathematical Monthly, Vol. 67, 349-353
%			(1960).
%			
%			\bibitem{reiner}
%			{ Reiner, David L.}
%			\textit{Enumeration in Music Theory},
%			The American Mathematical Monthly, Vol. 92, No. 1
%			(1985).
			
%			\subsubsection*{\autoref{ch:filosofia}}
%			
%			\bibitem{basomba}
%			{ Basomba Garc\'ia, Daniel.} 
%			\textit{El \'ultimo Bach y el dodecafonismo como ideal musical: una lectura est\'etica y sociol\'ogica},
%			Universidad Carlos III de Madrid.
%			Tesis Doctoral en Ciencia Pol\'itica y Sociolog\'ia
%			(2013)
			
%			NADA
%			\bibitem{bhalerao}
%			{ Bhalerao, Rasika.} 
%			\textit{The Twelve-Tone Method of Composition},
%			Math 336.
%			Prof. Jim Morrow
%			(2015)
			
%			\bibitem{morris}
%			{ Morris, Robert.}
%			\textit{Mathematics and the Twelve-Tone System: Past, Present, and Future},
%			Perspectives of New Music 45.2 
%			(2007)
			
%			\bibitem{cook}
%			{ Cook, Nicholas.} Chapter 9: ``Analyzing Serial Music'',
%			\textit{A Guide to Musical Analysis},
%			New York: G. Braziller
%			(1987)

%			\bibitem{roberts}
%			{ Roberts, Gareth E.}
%			\textit{Composing with Numbers: Arnold Schoenberg and His Twelve-Tone Method},
%			Math/Music: Aesthetic Links
%			(2012).
						
	\end{thebibliography}