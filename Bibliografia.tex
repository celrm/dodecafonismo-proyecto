	\begin{thebibliography}{00}
		\addcontentsline{toc}{chapter}{Bibliografía}
			
			\bibitem{wright}
			{\sc Wright, David.} 
			\textit{Mathematics and Music},
			American Mathematical Society 
			(2009).			
			
			\subsubsection*{\autoref{ch:historia}}
			
			\bibitem{kinney}
			{\sc Kinney, James P.} 
			\textit{Twelve-tone Serialism: Exploring the Works of Anton Webern},
			Undergraduate Honors Theses.
			Paper 1
			(2015)
			
			\bibitem{diaz}
			{\sc Díaz de la Fuente, Alicia.} 
			\textit{Estructura y significado en la música serial y aleatoria},
			Universidad Nacional de Educación a Distancia.
			Tesis Doctoral en Filosofía
			(2005)			
			
			\subsubsection*{\autoref{ch:dodecafonismo}}
			
			\bibitem{delgado}
			{\sc Delgado García, Fernando.} 
			Clases y material de Historia de la Música, 5$^{\circ}$ y 6$^{\circ}$ de Enseñanzas Profesionales del Conservatorio Profesional de Música Arturo Soria, cursos 2014-15 y 2015-16.
			
			\bibitem{boulez}
			{\sc Boulez, Pierre.}
			\textit{Schoenberg is dead},
			The Score
			(1952).
			
			\bibitem{dominguez}
			{\sc Domínguez Romero, Manuel.} 
			\textit{Las Matemáticas en el Serialismo Musical},
			Sigma n.24 
			(2004).

			\subsubsection*{\autoref{ch:suite}}
			
			\bibitem{ilomaki}
			{\sc Ilom\"aki, Tuukka.}
			\textit{On the Similarity of Twelve-Tone Rows},
			Sibelius Academy
			(2008).
			
			\bibitem{hyde}
			{\sc Hyde, Martha.} Chapter 4: “Dodecaphonism: Schoenberg”,
			\textit{Models of Musical Analysis: Early Twentieth-century Music},
			Ed. Mark Everist and Jonathan Dunsby.
			Oxford: Blackwell
			(1993)
					
			\bibitem{xiao}
			{\sc Xiao, June.} 
			\textit{Bach’s Influences in the Piano Music of Four 20th Century Composers},
			Indiana University Jacobs School of Music.
			Doctoral Theses in Music
			(2014)
			
			\bibitem{clercq}
			{\sc Clercq, Trevor de.} 
			\textit{A Window into Tonality via the Structure of Schoenberg's ``Musette'' from the Piano Suite, op. 25},
			Theory/Analysis of 20th-Century Music.
			Prof. David Headlam
			(2006)
			
			\subsubsection*{\autoref{ch:grupo}}			
			
			\bibitem{hunter}
			{\sc Hunter, David J.; von Hippela, Paul T.}
			\textit{How Rare Is Symmetry in Musical 12-Tone Rows?},
			The American Mathematical Monthly, Vol. 110, No. 2
			(2003).
			
			\subsubsection*{\autoref{ch:acciones}}
			
			\bibitem{armstrong}
			{\sc Armstrong, M. A.} Chapter 6: “Permutations”, Chapter 17: “Actions, Orbits, and Stabilizers”, Chapter 18: “Counting Orbits”,
			\textit{Groups and Symmetry},
			New York: Springer-Verlag
			(1988)
			
			\subsubsection*{\autoref{ch:espectros}}
			
			\bibitem{polygons}
			{\sc Golomb, S. W., Welch, L. R.}
			\textit{On the enumeration of polygons},
			The American Mathematical Monthly, Vol. 67, 349-353
			(1960).
			
			\bibitem{reiner}
			{\sc Reiner, David L.}
			\textit{Enumeration in Music Theory},
			The American Mathematical Monthly, Vol. 92, No. 1
			(1985).
			
%			\subsubsection*{\autoref{ch:filosofia}}
%			
%			\bibitem{basomba}
%			{\sc Basomba García, Daniel.} 
%			\textit{El último Bach y el dodecafonismo como ideal musical: una lectura estética y sociológica},
%			Universidad Carlos III de Madrid.
%			Tesis Doctoral en Ciencia Política y Sociología
%			(2013)
			
%			NADA
%			\bibitem{bhalerao}
%			{\sc Bhalerao, Rasika.} 
%			\textit{The Twelve-Tone Method of Composition},
%			Math 336.
%			Prof. Jim Morrow
%			(2015)
			
%			\bibitem{morris}
%			{\sc Morris, Robert.}
%			\textit{Mathematics and the Twelve-Tone System: Past, Present, and Future},
%			Perspectives of New Music 45.2 
%			(2007)
			
%			\bibitem{cook}
%			{\sc Cook, Nicholas.} Chapter 9: “Analyzing Serial Music”,
%			\textit{A Guide to Musical Analysis},
%			New York: G. Braziller
%			(1987)

%			\bibitem{roberts}
%			{\sc Roberts, Gareth E.}
%			\textit{Composing with Numbers: Arnold Schoenberg and His Twelve-Tone Method},
%			Math/Music: Aesthetic Links
%			(2012).
						
	\end{thebibliography}